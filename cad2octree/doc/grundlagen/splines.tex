% grundlagen/splines.tex
% B-Splines
%
% Ausarbeitung zur Diplomarbeit Nr. 2035 - "Erzeugung und Evaluierung 
% von Oktalbaumstrukturen als Schnittstelle zu CAD-Programmen"
%
% Autor: Stefan Mahler 2002
%   Universitaet Stuttgart, SgS
% Betreuer: Ralf Mundani

\subsubsection{B-Splines}
\label{spline}
%-De-Boor-Algorithmus
%-B-Spline-Funktionen
%-B-Spline-Kurven
B-Spline-Kurven vom Grad $p$ sind definiert durch
\gl{gl_b_spline_kurve}{{\bf x}(u) = \sum_{i=0}^{n}~ N_i^p(u)~
    { \bf c}_i\mbox{,}}
wobei ${ \bf c}_i$ die Kontrollpunkte und $N_i^p$ die zur B-Spline-Kurve 
geh"origen Basisfunktionen "uber den Knotenvektor $U = \{u_0, \dots, u_n\}$ 
darstellen. Analog zu den zusammengesetzten \bez-Kurven ist $u_{\min} = u_0 < 
u_1 < \dots < u_n = u_{\max}$ eine Zerlegung des Intervalls $[u_{\min}, 
u_{\max}]$. 
Wieder ergeben sich Teilkurven zu jedem Teilintervall. Die Glattheit der 
Gesamtkurve ist hier jedoch auch "uber die Sto"sstellen der Teilkurven 
gesichert, da B-Spline-Basisfunktionen anstelle der Bernsteinpolynome 
verwendet werden. 

Die Basisfunktionen $p$-ter Ordnung $N_i^p$ k"onnen ausgehend von der 
Basisfunktion erster Ordnung $N_i^1$
\gl{gl_basis_fkt_lin}{%
    N_i^1(u) = \left\{
    \begin{array}{rl}
	1 & u_i \le u < u_{i+1}\\
	0 & \mbox{sonst}~
    \end{array} \right.
    ~(0 \le i < n)}
rekursiv mit
\gl{gl_basis_fkt_rek}{N_i^p(u) = {{u - u_i}\over{u_{i+p-1} - u_i}}~N_i^{p-1}(u) 
    + {{u_{i+p} - u}\over{u_{i+p} - u_{i+1}}}}
berechnet werden. 
Alle Terme $(u_i-u)/(u_{i+p-1}-u_i)$ und $(u_{i+p}-u)/(u_{i+p}-u_{i+1})$ 
werden f"ur $u_{i+p-1}=u_i$ bzw. $u_{i+p}=u_{i+1}$ mit $0$ festgelegt.
Somit gilt 
\gl{gl_sum_basisfuns_eins}{\sum_{i=0}^{n-p} N_i^p(u) = 1\mbox{,}~
    ~\forall{} u \in [u_{p-1}, u_{n-p+1}]}
und
\gl{gl_basis_funs_null_gr_null}{N_i^p(u) \left\{
    \begin{array}{rl}
	> 0 & u \in (u_i, u_{i+p})\\
	= 0 & \mbox{sonst.}
    \end{array} \right.}

Mit Hilfe des \emph{de-Boor-Algorithmus} k"onnen B-Spline-Kurven beliebigen 
Grades ohne Kenntnis der Basisfunktionen erzeugt werden. Er ist das 
"Aquivalent des Algorithmus von Casteljau f"ur B-Spline-Kurven:
Sei eine B-Spline-Kurve $p$-ten Grades durch die Kontrollpunkte 
${\bf c}_0, \dots, {\bf c}_n$ mit zugeordneten Parameterwerten 
$u_0, \dots, u_n$, wobei $u_{\min} = u_0 < \dots < u_n = u_{\max}$ ist, 
gegeben. 

Setzt man 
${\bf b}_i^{(0)} = \cases{ {\bf c}_i & i \in [0, n]\\
			   0	     & \mbox{sonst,} }$
dann enth"alt ${\bf b}_{\phi}^{(p-1)}(u)$ nach $p-1$-maliger Anwendung der 
Rekursionsformel
\gl{gl_de_boor}{{\bf b}_i^{(r)}(u) = {{u_{i+p-r} - u}\over{u_{i+p-r} - u_i}}~
    { \bf b}_{i-1}^{(r-1)}(u) + {{u - u_i}\over{u_{i+p-r} - u_i}}~
    { \bf b}_{i}^{(r-1)}(u)}
den Punkt der B-Spline-Kurve zu $u$ mit $u \in [u_{\phi}, u_{\phi+1}]$ und 
$0 \le \phi < n$.
Substituiert man 
\gl{gl_subst_u_term}{\alpha_i^r = {{u - u_i}\over{u_{i+p-r} - u_i}}}
erh"alt man die zum Algorithmus von Casteljau analoge Darstellung
\gl{gl_de_boor_Subst}{{ \bf b}_i^{(r)} = (1 - \alpha_i^r)~
    { \bf b}_{i-1}^{(r-1)} + \alpha_i^r~{ \bf b}_i^{(r-1)}\mbox{.}}
Mit $p = n + 1$, dem Knotenvektor $U = \{0, \underbrace{1, \dots, 1}_{n}\}$ 
und $u_{-n} = \dots = u_{-1} = u_{\min}$ ergibt sich $\alpha_i^r = u$.
\bez-Kurven sind somit nur ein Spezialfall von B-Spline-Kurven.

B-Splines besitzten viele Eigenschaften von \bez-Kurven 
(vgl. Seite \pageref{bez_props}):
\begin{itemize}
\item Konvexe-H"ullen-Eigenschaft (sogar in strengerer Form)
\item Invarianz bez"uglich affiner Transformationen 
\item Endpunkt-Interpolation, falls die ersten bzw. letzten $p+1$ Knoten den 
    gleichen Wert besitzen
\item Kontrollpunkt-Symmetrie
\item Formerhaltungs-Eigenschaft 
\item Beschr"ankte Schwankung (variation diminishing)
\end{itemize}

Hinzu kommt die lokale Tr"agereigenschaft 
(vgl. \glref{gl_basis_funs_null_gr_null}): 
Modifikationen an einem Kontrollpunkt bleiben auf $p$ Nachbarn (bei einer 
B-Spline-Kurve $p$-ten Grades) begrenzt.

Allerdings sind B-Splines wie \bez-Kurven \emph{nicht} invariant bez"uglich 
projektiven Abbildungen. Diesen Nachteil kann man durch NURBS beseitigen.

%% End of Document
