% grundlagen/nurbs.tex
% NURBS
%
% Ausarbeitung zur Diplomarbeit Nr. 2035 - "Erzeugung und Evaluierung 
% von Oktalbaumstrukturen als Schnittstelle zu CAD-Programmen"
%
% Autor: Stefan Mahler 2002
%   Universitaet Stuttgart, SgS
% Betreuer: Ralf Mundani

\subsubsection{NURBS}
\label{nurbs}
%- projektive Invarianz
%- Einf"uhren von Gewichten
NURBS ist die Kurzform f"ur \emph{Non Uniform Rational B-Splines}. 
Wie schon aus den Namen hervorgeht, werden hier im Gegensatz zu \bez-Kurven 
und Splines rationale Funktionen zur Beschreibung verwendet. 
NURBS sind definiert durch
\gl{gl_nurbs}{{ \bf x}(u) = {\scriptstyle{\sum^n_{i=0}} ~ N_i^p(u) ~ w_i 
    ~ { \bf c}_i \over \scriptstyle{\sum^n_{i=0}} ~ N_i^p(u) ~ w_i}\mbox{.}}
Den Kontrollpunkten ${\bf c}_i$ werden noch zus"atzlich Gewichte $w_i$ 
zugeordnet, die "uberlicherweise als Formparameter verwendet werden.
Je gr"o"ser das Gewicht eines Kontrollpunkts ist, desto st"arker ist sein 
Einfluss auf den Kurvenverlauf, indem sich die NURBS-Kurve st"arker diesem 
Kontrollpunkt ann"ahert.

Werden allerdings alle Gewichte um den gleichen Faktor ver"andert, "andert das 
nichts am Verlauf der NURBS-Kurve. Besitzen alle Kontrollpunkte das gleiche 
Gewicht, so ergibt das wieder die 'gew"ohnliche' B-Spline. B-Splines (und 
somit auch \bez-Kurven) sind also Spezialf"alle von NURBS.

Eine NURBS-Kurve kann man sich als eine in den 3-dimesionalen Raum projezierte 
4-dimensionale B-Spline-Kurve vorstellen.
Wie sich zeigen l"asst, sind NURBS-Kurven bez"uglich projektiven Abbildungen 
invariant.

\bez-Kurven, B-Splines und NURBS besitzen somit i.A. unterschiedliche 
Eigenschaften. Andererseits sind ihre Datenstrukturen unterschiedlich komplex. 
Es existieren Algorithmen zum Konvertieren geometrischer Modelle von 
NURBS-Darstellung in \bez-Kurven. Die drei Darstellungsformen \bez-Kurven, 
B-Splines und NURBS sind somit gleichm"achtig. Je nach Aufgabenstellung 
kann es evtl. g"unstig sein, vor der Bearbeitung des Modells die eine in eine 
andere Darstellungsform umzuwandeln.

%% End of Document
