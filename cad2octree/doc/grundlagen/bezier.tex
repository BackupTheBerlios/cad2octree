% grundlagen/bezier.tex
% Bezier-Kurven
%
% Ausarbeitung zur Diplomarbeit Nr. 2035 - "Erzeugung und Evaluierung 
% von Oktalbaumstrukturen als Schnittstelle zu CAD-Programmen"
%
% Autor: Stefan Mahler 2002
%   Universitaet Stuttgart, SgS
% Betreuer: Ralf Mundani

\subsubsection{\bez-Kurven}
\label{bezier}
%-Bernstein-Polynome
%-Casteljau-Algorithmus
%-Zusammengesetzte Bezier-Kurven

Eine Kurve, die durch 
\gl{gl_bezier_kurve}{{\bf x}(t) = \sum_{i=0}^n ~B_i^n(t)~{ \bf b}_i\mbox{, }~
    ~t \in{} [0, 1]}
beschrieben wird, wird als \bez-Kurve bezeichnet. Dabei sind die 
Bernstein-Polynome $ B_i^n $ definiert durch
\gl{gl_bernstein_2d}{B_i^n(t) = \binom{n}{i} ~(1 - t)^{n-i}~t^i\mbox{, }~ 
    ~t \in{} [0, 1]\mbox{, }~i= 0,\dots,n\mbox{.}}
Damit gilt
\gl{gl_sum_bernstein_eins}{\sum_{i=0}^n B_i^n(t) = 1\mbox{, }~
    ~\forall{} t \in [0, 1]}
und 
\gl{gl_bernstein_rand_a}{B_0^n(0) = B_n^n(1) = 1} bzw.
\gl{gl_bernstein_rand_b}{B_0^n(1) = B_n^n(0) = 0\mbox{.}}

\bez-Kurven sind die einfachste Form einer symmetrischen Darstellung 
durch polynomielle Approximation mit Kontrollpunkten. Die 
Kontrollpunkte ${\bf b}_i$ einer \bez-Kurve werden \bez-Punkte genannt. 
Sie bilden ein Kontrollpolygon.
\bez-Kurven besitzen folgende wichtige Eigenschaften: \label{bez_props}
\begin{enumerate}
\newcommand\nritem[2]{\item \textbf{#1:} \label{#2}}
\nritem{Konvexe-H"ullen-Eigenschaft}{bez_prop_convex_hull}
    Eine \bez-Kurve liegt in der konvexen H"ulle ihres Kontrollpolygons.
\nritem{Stetigkeit, Differenzierbarkeit}{bez_prop_diff}
    \bez-Kurven sind stetig. Besitzt die \bez-Kurve Mehrfachkontrollpunkte 
    oder besitzt der die Kontrollpunkte miteinander verbindende Kantenzug 
    Schleifen (das Kontrollpolygon ist also nicht konvex), kann die Kurve 
    jedoch an einigen Stellen eventuell nicht differenzierbar sein.
\nritem{Interpolation der Kurvenenden}{bez_prop_intpol_kurvend}
    Kontrollpolygon und \bez-Kurve stimmen in Anfangs- und Endpunkt 
    "uberein. Die inneren \bez-Punkte werden i.A. nur approximiert und 
    liegen somit dann auch nicht auf der \bez-Kurve.
\nritem{Kurvenanstieg an den Kurvenenden}{bez_prop_diff_kuvend}
    Die Strecken $\overline{{\bf b}_0{\bf b}_1}$ und 
    $\overline{{\bf b}_{n-1}{\bf b}_n}$ des Kontrollpolygons verlaufen im 
    Anfangs- und Endpunkt tangential zur \bez-Kurve.
\nritem{Einfluss von Kontrollpunkten, Globale Modifikation}{bez_prop_cp_einfl}
    ${\bf b}_i$ hat den gr"o"sten Einflu"s auf ${\bf x}(t)$ bei $t = i/n$. 
    Bis auf Anfangs- und Endpunkt der Bezierkurve werden zur Berechung 
    eines Kurvenpunkts \emph{alle} Kontrollpunkte (wenn auch in 
    unterschiedlichem Ausma"s) ber"ucksichtigt.
    Die Modifikation eines Kontrollpunktes wirkt sich global auf die
    \bez-Kurve aus.
\nritem{Formerhaltungs-Eigenschaft}{bez_prop_formerh}
    Ist das Kontrollpolygon konvex, so ist es auch die \bez-Kurve.
\nritem{Beschr"ankte Schwankung}{bez_prop_beschr_schw}(auch \emph{variation
    diminishing} genannt)
    Keine Gerade schneidet die \bez-Kurve "ofter als das entsprechende 
    Kontrollpolygon.
    Daraus ist auch direkt die Eigenschaft der \emph{linearen Pr"azision} 
    ableitbar: Liegen die Kontrollpunkte ${ \bf b}_0, \dots, { \bf b}_n$ einer 
    \bez-Kurve kollinear, dann liegt die \bez-Kurve auf der Strecke 
    $\overline{{\bf b}_0{\bf b}_n}$.
\nritem{Graderh"ohung}{bez_prop_grad_incr}
    Bei fortgesetzter Graderh"ohung durch Einf"ugen zus"atzlicher 
    Kontrollpunkte konvergiert das Kontrollpolygon gegen die \bez-Kurve.
\nritem{Affine Invarianz}{bez_prop_aff_inv}
    \bez-Kurven sind bez"uglich Rotation, Skalierung ($\not= 0$), Spiegelung 
    und Translation invariant.
\end{enumerate}

Eine effiziente und numerisch stabile M"oglichkeit zur Berechnung der 
Kurvenpunkte ${ \bf x}(t)$ ist der \emph{Algorithmus von de Casteljau}. 
Abbildung \ref{abb_alg_casteljau} zeigt ein Schema zur geometrischen 
Konstruktion von ${\bf x}(t=\frac{2}{3})$ f"ur $n=3$ nach diesem Algorithmus. 

\graf{abb_alg_casteljau}{Algorithmus von de Casteljau (\emph{geom.})}{%
    casteljau}

Die Eigenschaften \ref{bez_prop_diff} und \ref{bez_prop_cp_einfl} k"onnen sich 
nachteilig auf die Modellierung von \bez-Kurven auswirken. Diese Nachteile 
wie mangelnde Diffenrenzierbarkeit und unerw"unschte Ausgleichseffekte von 
Kontrollpunkten infolge fehlender Lokalit"at der Kontrollpunktwirkung werden 
h"aufig bei h"ohergradigen \bez-Kurven sichtbar. Zudem sind mathematische 
Beschreibungen der \bez-Kurven vom Grad $\ge 10$ unhandlich, Berechnungen sind 
rechenzeitintensiv. Andereseits erfordert das Nachbilden komplexer realer 
Kurven evtl. viele Kontrollpunkte. 

Einen Ausweg bieten \textbf{zusammengesetzte \bez-Kurven}. Die Gesamtkurve 
wird aus einzelnen Kurvensegmenten gleichen Grads zusammengesetzt. 
Das Segment $i$ der \bez-Kurve besteht somit aus den Kontrollpunkten 
${\bf b}_{0,i}, \dots, {\bf b}_{p, i}$, wenn die Segmente vom Grad $p$ sind. 
Nun wird zweckm"a"siger Weise ein Verfahren ben"otigt, was zu einem globalen 
Parameter $u$ eindeutig den lokalen Parameter $t$ des $i$-ten Segments 
bestimmt. O.B.d.A. sei angenommen, dass die Gesamtkurve normalisiert, also 
$u \in [0, 1]$, sei. L"auft $u$ somit "uber $[0, 1]$ wird die Gesamtkurve 
"uberstrichen. Mit Hilfe der Zerlegung $0=u_0<u_1<\dots<u_n<u_{n+1}=1$ des 
Intervalls $[0, 1]$ kann das $i$-te Segment "uber dem Intervall 
$I_i = [u_i, u_{i+1}]$ definiert werden. Der lokale Paramterwert $t$ des 
Intervalls $I_i$ kann dann mit
\gl{gl_def_t_durch_u}{t = {{u - u_i} \over {u_{i+1} - u_i}}\mbox{, }~
    ~u \in{} [u_i, u_{i+1}]}
definiert werden. 

Wegen Eigenschaft \ref{bez_prop_intpol_kurvend} l"asst 
sich die Stetigkeit von zusammengesetzten \bez-Kurven leicht bewerkstelligen: 
Hierf"ur m"ussen die Anfangspunkte des nachfolgenden Segments nur mit den 
jeweiligen Endunkten des Vorg"angersegments "ubereinstimmen:
\gl{gl_bez_zusammenges_cnt}{{ \bf b}_{n, i} = { \bf b}_{0, i+1}\mbox{.}} 
Schwieriger l"asst sich allerdings hinreichende Differenzierbarkeit an den 
Segment"uberg"angen gew"ahrleisten. Um an den Segment"uberg"angen 
\conti-Stetigkeit zu erhalten, muss
\gl{gl_bez_picewise_diff}{
    {{{ \bf b}_{n, i} - { \bf b}_{n-1, i}} \over {u_{i+1} - u_i}} 
    = {{{ \bf b}_{1, i+1} - { \bf b}_{n, i}} \over {u_{i+2} - u_{i+1}}}}
gelten. Dies bedeutet, dass die drei Punkte ${\bf b}_{n-1, i}$, 
${\bf b}_{n, i}={\bf b}_{0, i+1}$ und ${\bf b}_{1, i+1}$ kollinear sein 
m"ussen \emph{und} dass ${\bf b}_{n, i}$ die Strecke 
$\overline{{\bf b}_{n-1, i}{\bf b}_{1, i+1}}$ im Verh"altnis 
\hbox{\((u_{i+1} - u_i) : (u_{i+2} - u_{i+1})\)} teilt. 
Um die h"aufig geforderte \contii-Stetigkeit zu erhalten, muss das Schema 
unter Einbeziehung der Punkte ${\bf b}_{n-2, i}$ und ${\bf b}_{2, i+1}$ 
entsprechend erweitert werden. 

Nachteilig an zusammengesetzten \bez-Kurven ist, dass die ersten und letzten 
Kontrollpunkte eines jeden Segments die Bedingungen 
\glref{gl_bez_zusammenges_cnt} und \glref{gl_bez_picewise_diff} erf"ullen 
m"ussen, um die Differenzierbarkeit der Gesamtkurve sicherzustellen. Daf"ur 
m"ussen eventuell zus"atzliche Kontrollpunkte eingef"ugt werden. Dieses 
Problem kann durch die Nutzung von B-Spline-Kurven umgangen werden.

%% End of Document
