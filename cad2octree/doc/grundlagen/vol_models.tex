% grundlagen/vol_models.tex
% Volumenmodelle
%
% Ausarbeitung zur Diplomarbeit Nr. 2035 - "Erzeugung und Evaluierung 
% von Oktalbaumstrukturen als Schnittstelle zu CAD-Programmen"
%
% Autor: Stefan Mahler 2002
%   Universitaet Stuttgart, SgS
% Betreuer: Ralf Mundani

\subsection{Volumenmodelle}
\label{volmodels}
Mit Hilfe von Volumenmodellen werden direkt die zu den beschreibenden 
K"orpern geh"orenden Volumina definiert. Prinzipiell sind hierf"ur zwei 
Wege m"oglich:
\begin{enumerate}
\item \label{raumpart_volmodel} Der Raum wird diskretisiert in Zellen 
    aufgeteilt und die einzelnen Zellen als zum K"orper zugeh"orig oder 
    nicht definiert oder
\item \label{konstr_volmodel} der K"orper wird konstruktiv aus Primitiven 
    aufgebaut. Die Primitive k"onnen eventuell parametrisierbar sein.
\end{enumerate}
Modelle des Typ \ref{raumpart_volmodel} werden auch raumpartitionierende 
Modelle genannt. Beispiele hierf"ur sind das Normzellen-Aufz"ahlungsschema 
oder das Oktalbaumschema. Hauptvorteil dieser Schemata ist, dass die 
Lokalisation eines Punktes (Befindet sich der Punkt innerhalb oder 
au"serhalb des K"orpers?) direkt aus dem Modell ermittelt werden kann und 
nicht hierf"ur aufw"andige Berechnungen erfolgen m"ussen. Dies ist gerade 
im Bereich der Simulation wichtig. Ein Vertreter des Typ 
\ref{konstr_volmodel} ist das CSG-Schema, welches im Bereich des CAD 
Anwendung findet.

\subsubsection[Normzellen-Aufz"ahlungsschema]{%
    Normzellen-Aufz"ahlungsschema (NAZ)}
\label{normzellen}
{
\let\mywidth=\linewidth
\parbox[t]{0.7\mywidth}{%
    Das zu betrachtende dreidimensionale endliche Teilgebiet wird vollst"andig 
    in gleich gro"se dreidimensionale Zellen, meistens W"urfel aufgeteilt. 
    Jede Zelle wird nun nach festgelegten Regeln als zum K"orper geh"orend 
    oder nicht geh"orend definiert. F"ur den K"orperrand findet h"aufig 
    folgende Regel Anwendung: Wird die Zelle von der K"orperoberfl"ache 
    geschnitten, so wird dem K"orper zugeordnet, wenn der Zellmittelpunkt sich 
    innerhalb des K"orpers befindet, ansonsten nicht. Abb. \ref{abb_naz} 
    zeigt ein Normzellen-Aufz"ahlungsschema im 2-Dimensionalen. 
}
\hspace{1em}
\parbox[t][4.5cm][c]{0.25\mywidth}{\bild{\linewidth}{abb_naz}{%
    NAZ}{naz}}
F"ur eine gegebene Zellgr"o"se ist die Pr"asentation des K"orpers durch 
Normzellen somit eindeutig. Je kleiner die Zellgr"o"se
desto genauer kann der K"orper dargestellt werden. Als Datenstruktur wird 
im dreidimensionalen Fall am einfachsten ein dreidimensionales Feld verwendet. 
Eine Verdopplung der Aufl"osung in alle Raumrichtungen hat damit aber auch 
eine Verachtfachung des Speicheraufwands zur Folge. 
Da aus dem Schema nicht direkt ersichtlich ist, welche Zellen den Rand 
eines Objekts widerspiegeln und sich in einem solcen Fall die Gr"o"se 
aller Zellen sich "andert, m"ussen alle Zellen f"ur eine 
verfeinerte Darstellung neu ermittelt werden.
Modelle in Normzellen-Aufz"ahlungsschemata werden h"aufig zur Aufbewahrung 
komprimiert.

\subsubsection{Zellzerlegungsschema}
{
\let\mywidth=\linewidth
\parbox[t]{0.5\mywidth}{%
    Wird nicht wie beim Normzellen-Aufz"ahlungsschema nur ein Primitiv 
    verwendet, sondern k"onnen unterschiedliche Zellarten genutzt werden, 
    spricht man allgemein vom Zellzerlegungsschema. 
}
\hspace{1em}
\parbox[t][2cm][c]{0.45\mywidth}{\bild{\linewidth}{abb_zellzerl}{%
    Mehrdeutige Darstellung beim Zellzerlegungsschema}{zellzerl}}
}

Neben Zellen mit glatter Oberfl"ache k"onnen auch Zellen mit gekr"ummter 
Oberfl"ache als Basisobjekte zum Einsatz kommen. 
Entsprechend dem Lego-Prinzip wird der K"orper aus den Basisobjekten, 
den Primitiven, zusammengesetzt. 

Wie Abb. \ref{abb_zellzerl} zeigt, muss die Pr"asentation des K"orpers nicht 
eindeutig sein, da unterschiedliche Primitive und in unterschiedlicher 
Reihenfolge zur Konstruktion des Objekts verwendet werden k"onnen.  

\subsubsection{Constructive Solid Geometry}
{
\let\mywidth=\linewidth
\parbox[t]{0.37\mywidth}{%
    Bei der Constructive Solid Geometry - kurz CSG-Schema - wird der 
    modellierte K"orper als Konstruktion durch Nutzung von bin"aren 
    Mengenoperationen auf Basisobjekte dargestellt. 
    Verwendbare Basisobjekte, auch Primitive genannt, sind z.B. Quader, 
    Kugeln, Kegel und Zylinder, die zuvor definiert wurden. 
}
\hspace{1em}
\parbox[t][4.5cm][c]{0.58\mywidth}{\begin{center}
    \bild{0.9\linewidth}{abb_csg}{CSG-Schema mit Konstruktionsbaum}{csg}
    \end{center}}
}

Typische Mengenoperationen sind Schnitt, Vereinigung und Differenz.
Als Erweiterung des Schemas k"onnen noch zus"atzlich die Transformationen 
Rotation, Translation bzw. Skalierung als Operationen neben 
Mengenoperationen zugelassen werden. 

Ein CSG-Schema kann durch einen Bin"arbaum dargestellt werden (vgl. 
Abb. \ref{abb_csg}). 
Dabei repr"asentieren die Bl"atter die verwendeten Primitive, an den 
inneren Knoten werden die genutzten Operationen vermerkt. 
Von den Bl"attern ausgehend wird mit Hilfe der Operationen sukzessive 
der K"orper konstruiert. 
Um eine eindeutige Darstellung des K"orpers zu erhalten, muss der 
Konstruktionsbaum normalisiert werden. Als Normalform k"onnen die disjunktive
oder konjunktive Normalform (DNF bzw. KNF) auftreten. 

\subsubsection{Verschiebegeometrieschema}
\label{vschiebgeom}
{
\let\mywidth=\linewidth
\parbox[t]{0.62\mywidth}{%
Beim Verschiebegeometrieschema wird eine endliche Fl"ache (z.B. Polygon)
entlang einer Verschiebekurve bewegt. 
Das "uberstrichene Gebiet wird als das Volumen des modellierten K"orpers 
aufgefasst. Wird z.B. ein Rechteck einen Vektor
entlang verschoben, dessen Richtung senkrecht zur Rechtecksfl"ache verl"auft,
entsteht ein Quader. 
Verschiebt man das Rechteck entlang einer Kurve, die
eine Kreislinie darstellt, wird ein (Hohl)Zylinder erzeugt (vgl. Abb.  
\ref{abb_verschgeo}).
}
\hspace{1em}
\parbox[t][4cm][c]{0.33\mywidth}{\bild{\linewidth}{abb_verschgeo}{%
    Ver\-schiebe\-geometrie\-schema}{verschgeo}}
}

\subsubsection{Primitiven-Instanzierung}
{
\let\mywidth=\linewidth
\parbox[t]{0.6\mywidth}{%
    Bei der Primitiven-Instanzierung werden alle K"orper aus einem 
    parametrisierbaren Grundobjekt erzeugt. Das Grundobjekt kann dabei 
    durchaus komplex sein. Ist das Grundprimiv z.B. eine Metallschraube mit 
    den Parametern Steigung, L"ange und Kopfgr"o"se, k"onnen verschiedene 
    Metallschraubenarten beschrieben werden. In Abb.  
    \ref{abb_primitivinst} ist ein 6-etagiges Geb"aude mit  
    6 Fensterspalten, das aus dem Primitiv 
    'Blockhaus' erzeugt wurde.
}
\hspace{1em}
\parbox[t][4cm][c]{0.35\mywidth}{\bild{\linewidth}{abb_primitivinst}{%
    Primitiven"=Instanzierung}{primitivinst}}


\subsubsection{Oktalb"aume}
\label{einf_octree}
{
\let\mywidth=\linewidth
\parbox[t]{0.65\mywidth}{%
    Beim Oktalbaumschema handelt es sich um ein hierachisch, rekursives 
    Schema, bei dem der zu diskretisierende Raum in Zellen unterteilt wird. 
    Es wird von einem, den gesamten K"orper umschlie"senden W"urfel 
    (boundary cube) ausgegangen. 

    Beginnend mit diesem W"urfel erfolgt eine sukzessive 
    Zerlegung bis der betrachtete W"urfel sich vollst"andig innerhalb oder 
    au"serhalb des K"orpers befindet oder die zuvor festgelegte Aufl"osung 
    den W"urfelabmessungen entspricht (dies ergibt sich direkt aus der 
    maximalen Anzahl der Zerlegungsschritte, welche auch der maximalen 
    Baumtiefe entspricht).
    Bei jedem Zerlegungsschritt wird der W"urfel in jede Achsrichtung 
    halbiert, woraus sich $8$ Unterw"urfel (Oktanten) ergeben. 
}
\hspace{1em}
\parbox[t][6cm][c]{0.3\mywidth}{\bild{\linewidth}{abb_quadtree}{%
    Raum\-partio\-nie\-rung durch Quadtree}{quadtree}}

Die Bl"atter werden als im K"orper befindlich oder nicht zum K"orper 
geh"orend markiert. 
F"ur Bl"atter, die Zellen repr"asentieren, die den K"orper teilweise 
enthalten, kann eine spezielle Markierung definiert werden. 
Alternativ werden solche Zellen je nach Anwendungsszenario entsprechend dem 
Normzellen-Aufz"ahlungsschema einfach zum K"orper mitgerechnet oder als 
au"serhalb des K"orpers liegend aufgefasst. Abb. \ref{abb_quadtree} 
zeigt einen Quadtree, das 2-dimensionale Pendant des Oktalbaums. 

Jedes Normzellen-Aufz"ahlungsschema mit W"urfeln als Primitive l"asst sich 
leicht in das entsprechende Oktalbaumschema "uberf"uhren und umgekehrt. 
Da nur die Randbereiche des K"orpers beim Oktalbaumschema maximal aufgel"ost 
werden m"ussen, ergibt sich im Allgemeinen ein signifikant geringerer 
Speicheraufwand als beim Normzellen-Aufz"ahlungsschema. Eine weitere Folge 
dieser Eigenschaft ist, dass bei jeder Verdopplung der Aufl"osungsgenauigkeit 
in jede Achsrichtung nur mit einer Vervierfachung des Speicheraufwands 
gerechnet werden muss\footnote{Diese Aussage ist nicht auf K"orper mit
fraktaler Struktur anwendbar. Solche Strukturen sind jedoch 
innerhalb des CAD eher untypisch.}. Zur persistenten Datenhaltung empfiehlt 
sich einer Linearisierung des Oktalbaums. Weitergehende Betrachtungen 
zu Eigenschaften von Oktalb"aumen finden sich im Abschnitt 
\ref{schnittstelle_octree}. Auf Oktalb"aumen arbeitende Algorithmen 
sind im Kapitel \ref{algo} dargestellt.

%% End of Document
