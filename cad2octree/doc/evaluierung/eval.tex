% evaluierung/eval.tex
% Dime als Importbibliothek
%
% Ausarbeitung zur Diplomarbeit Nr. 2035 - "Erzeugung und Evaluierung 
% von Oktalbaumstrukturen als Schnittstelle zu CAD-Programmen"
%
% Autor: Stefan Mahler 2002
%   Universitaet Stuttgart, SgS
% Betreuer: Ralf Mundani

\chapter{Unterst"utzte Formate}
\label{eval}
%Warum ? Bindeglied zwischen CAD und Simulation
%Importformat/-bibliothek,-format
%Anforderungen
%Eigenschaften
Wie bereits diskutiert, eignen sich Oktalb"aume zur Integration von 
Modellierung und Simulation. Durch die im Rahmen dieser Arbeit entstandene 
Implementierung sollen Oktalb"aume unter Zuhilfenahme eines 
Oberfl"achenmodells erzeugt werden k"onnen. 

Es musste ein \emph{geeignetes} Format f"ur das Oberfl"achenmodell festgelegt
werden, aus welchem der Oktalbaum generiert wird. Es wird im Folgenden als
Importformat bezeichnet. Abschnitt \ref{import} besch"aftigt sich
mit Auswahl und Eigenschaften des gew"ahlten Importformats.

Der erzeugte Oktalbaum wird abschlie"send in eine Datei geschrieben. 
Das hierf"ur verwendeteFormat wird als Exportformat bezeichnet. Abschnitt 
\ref{export_pot} erl"autertdas unterst"utzte Exportformat \emph{POT}.

\section{Import}
\label{import}
Um ein geeignetes Importformat finden zu k"onnen, m"ussen zun"achst Kriterien 
festgelegt werden, die dieses Format erf"ullen muss. Die Auswahlkriterien
sind Bestandteil des Abschnitts \ref{import_kriterien}. Im zweiten Schritt
erfolgt die Auswahl eines Formats nach diesen Kriterien. 

Zu dem gew"ahlten DXF-Format existieren frei zug"angliche Bibliotheken, die
den Umgang mit dem Format erleichtern. Abschnitt \ref{dxf_bibliotheken} 
bewertet drei DXF-Bibliotheken hinsichtlich ihrer Anwendbarkeit zur Umsetzung 
der Aufgabenstellung.

\subsection{Anforderungen an Importformate}
\label{import_kriterien}
Die Anforderungen ergeben sich unmittelbar aus der Aufgabenstellung f"ur die
gesamte Arbeit und sind:
\begin{description}
\item [Unterst"utzung von Dreiecksnetzen und Freiformfl"achen] ~\\ 
    Das Format muss Strukturen bereitstellen, die es erm"oglichen, sowohl 
    K"orper mit einfacher polygonaler Oberfl"ache (mindestens Drei- oder 
    Vierecke) als auch Freiformfl"achen in Form von Spline-Fl"achen zu 
    generieren.
    Es sollte auch eine gemischte Darstellung (eine Datei enth"alt z.B. zwei
    K"orper, von denen der eine mit triangulierter Oberfl"ache und der andere 
    durch Splinefl"achen modelliert ist) m"oglich sein.
\item [Erzeugbarkeit durch in der Abteilung vorhandene Applikationen] ~\\ 
    Die in der Abteilung SgS vorhandenen regul"aren 
    Fl"achenmodelle\footnote{Fl"achenmodelle von starren K"orpern 
    im Sinne der geometrischen Modellierung}, die in einem "ublichen 
    Format vorliegen,  sollen durch eine Exportfunktion, 
    die sich bereits in einem der in der Abteilung genutzten 
    Modellierungswerkzeuge befindet, einfach in das Importformat "uberf"uhrt 
    werden k"onnen, falls das Oberfl"achenmodell nicht bereits in diesem 
    Format vorliegt. Alternativ gibt es zu diesem Zweck ein frei verf"ugbares 
    Konvertierungswerkzeug. Das so erzeugte Modell stellt (nat"urlich) den 
    urspr"unglich modellierten K"orper in einer regul"aren Form da.
    Ein Modell soll auch mehrere K"orper enthalten k"onnen.
\item [Hohe Verbreitung] ~\\ 
    Das Format ist am Markt verbreitet. Es gibt 
    eine Vielzahl von Werkzeugen, die das Format verwenden. Es finden sich 
    Verweise auf das Format in der einschl"agigen Literatur. Es gibt 
    Anwendungen zum Erstellen und Bearbeiten von geometrischen Modellen, 
    die in diesem Format vorliegen. 
\item [Freie Zug"anglichkeit der Formatspezifikation] ~\\ 
    Die Formatspezifikation ist frei einsehbar. Es finden sich 
    Formatbeschreibungen im Internet und in einschl"agiger Literatur.
    Es handelt sich also um ein standardisiertes Format oder um einen offenen
    Pseudostandard.
\item [Einfaches Handling] ~\\ 
    Idealerweise existieren frei nutzbare Bibliotheken f"ur das Format. 
    Alternativ ist das Format einfach und leicht verst"andlich strukturiert. 
    Dies ist auch bei vorhandenen Bibliotheken f"ur Verifikationen
    w"unschenswert. Hierzu geh"oren auch frei verf"ugbare Betrachter f"ur 
    dieses Format. ASCII-Formate sind Bin"arformaten wegen der leichteren 
    Verifizierbarkeit vorzuziehen.
\end{description}
Die Formate DXF, IGES und STEP erf"ullen diese Kriterien. IGES und STEP 
werden von zahlreichen kommerziellen Systemen unterst"utzt. F"ur DXF finden 
sich hingegen auch viele frei verf"ugbare Quellen, weshalb die Wahl auf
DXF als Importformat fiel. Tabelle \ref{tbl_importformate} stellt die Formate
DXF, IGES und STEP gegen"uber. 

\tabbeg{}
\newcommand\headline[4]
    {\textbf{\emph{#1}} & \textbf{#2} & \textbf{#3} & \textbf{#4} \\}
\newcommand{\feature}[4]{\emph{#1} & #2 & #3 & #4 \\ \hline }
\begin{tabular}{|l|rrr|}
\hline
\headline{Format}{DXF}{IGES}{STEP}
\hline
\feature{Version}{2002}{5.3}{214}
\feature{Quelle}{\cite{dxf_ref}}{\cite{iges_docu}}{\cite{step_docu}}
\feature{Standard}{offener Psyeudostandard}{ANS US}{AP}
\end{tabular}
\tabend{tbl_importformate}{Formate DXF, IGES und STEP}

DXF besitzt einen relativ einfachen Aufbau. DXF ist ein ASCII-Format. Es gibt 
auch eine Bin"arversion DXB. Die Geometrieinformationen werden in der Sektion
\emph{Entity} abgespeichert. F"ur unterschiedliche geometrische Formen gibt 
es unterschiedliche Entities. F"ur diese Arbeit sind die Entities 3DFACE 
f"ur planare Oberfl"achen (Drei- oder Vierecke), POLYLINE (gekr"ummte Kurve 
oder Fl"ache, in diesem Fall kubische B-Spline-Fl"ache) und VERTEX (Einzelner 
Punkt der Spline-Fl"ache) wichtig. Einzelne Informationen finden sich hinter 
dem entsprechenden \emph{Group Code} wieder. So steht beispielsweise 
hinter dem Group Code 10 der 3DFACE-Entity die x-Koordinate des ersten 
Eckpunkts. Die Orientierung von Fl"achen ergibt sich aus der Reihenfolge 
der Eck-/Kontrollpunkte (vgl. mit \gvref{gl_flaechenorientierung}).
Die Entities werden einfach hintereinander gespeichert. 

\subsection{DXF-Bibliotheken}
\label{dxf_bibliotheken}
Im Folgenden wurden einige DXF-Bibliotheken evaluiert. Gesucht wurde eine 
Bibliothek zum Einlesen der im DXF-Format vorliegenden Geometriedaten. 

Folgende Kriterien wurden zur Bewertung der Importbibliotheken angewendetet:
\begin{description}
\item[Lizenz] ~\\ 
    Die Bibliothek muss frei verf"ugbar sein. Die hier betrachteten 
    Bibliotheken stehen unter (L)GPL und sind somit frei nutzbar.
\item[Installierbarkeit] ~\\ 
    Mit welchem Aufwand kann die Bibliothek zum Laufen gebracht werden?
\item[Funktionalit"at] ~\\ 
    Hierbei wurde getestet, inwieweit die Bibliotheken
    mit Geometriemodellen von K"orpern mit triangulierten bzw. 
    Splineoberfl"achen zurechtkamen. Hierf"ur konnten zum Teil beiliegende 
    Beispielprogramme verwendet werden. In anderen F"allen wurde die 
    eingeschr"ankte Benutzbarkeit aus der Dokumentation ersichtlich.
    Ein weiteres Augenmerk lag auf dem Einarbeitungsaufwand f"ur das
    Verst"andnis der relevanten Teile und der Komplexit"at der Einbindung.
    Die Bibliothek muss in C++-Programme integrierbar sein.
\item[Dokumentation] ~\\ 
    Mit welchen Mitteln wird die Bibliothek dokumentiert?
    Welchen Umfang hat die Dokumentation? Wie verst"andlich ist sie?
\item[Aktivit"at] ~\\ 
    Inwieweit wird die Bibliothek weiterentwickelt und gepflegt? 
    Wann fand die letzte Aktualisierung statt?
\end{description}

\tabbeg{}
\newcommand\headline[4]
    {\textbf{\emph{#1}} & \textbf{#2} & & \textbf{#3} & & \textbf{#4} \\}
\newcommand{\feature}[4]{\emph{#1} & #2 & & #3 & & #4 \\ \hline }
\begin{tabular}{|l|rcrcr|}
\hline
\headline{Library}{dime}{dxflib}{libdxf}
\hline
\feature{Version}{0.9.1}{0.1.2}{0.7}
\feature{Quelle}{\cite{dime}}{\cite{dxflib}}{\cite{libdxf}}
\feature{Lizenz}{GPL}{LGPL}{GPL}
\feature{Paket-Format}{tar.gz}{tar.gz}{tar.gz}
\feature{Installierbarkeit}{\neutral}{\neutral}{\neutral}
\feature{Funktionalit"at}{\positive}{\negative}{\neutral}
\feature{Dokumentation}{\neutral}{\negative}{\negative}
\feature{Aktivit"at}{\positive}{\positive}{\negative}
\feature{Verbreitung}{\positive}{\neutral}{\negative}
\feature{Eignung}{\positive}{\negative}{\negative}
\end{tabular}
\tabend{tbl_importlibs}{DXF-Bibliotheken dime, dxflib und libdxf}

Tabelle \ref{tbl_importlibs} zeigt die Bewertung f"ur dime, dxflib und libdxf.
Laut Dokumentation lassen sich alle 3 Bibliotheken sowohl auf Unix-Derivaten
wie auf allen Windows-Versionen einsetzen: Zur Installation werden lediglich 
neben einem \texttt{tar.gz}-Entpacker, ein \texttt{gcc}-kompatibler Compiler 
und eine funktionsf"ahige \texttt{make}-Umgebung ben"otigt. Die Bibliotheken 
wurden ausschlie"slich auf Linux (SuSE Linux 8.0) getestet.

Ein Vielzahl von Links verweisen auf die \textbf{dime}-Bibliothek. Sie 
l"asst sich u.a. als Debian-Paket auf den Debian-Seiten finden. Die hier 
eingesetzte Version stammt von Coin3d, dem Hersteller von dime. Zuerst muss 
das \texttt{tar.gz}-Archiv entpackt werden. Anschlie"send kann mit Hilfe des 
\texttt{configure}-Werkzeugs und \texttt{make} dime relativ einfach 
installiert werden. Neben dem Source-Code ist eine doxygen\footnote{Mit Hilfe 
von doxygen (\cite{doxygen}) kann aus Beschreibungstexten u.a. im 
C++-Source-Code eine Klassenreferenz ("ahnlich \texttt{javadoc} f"ur Java) 
generiert werden.}-generierte 
Klassenreferenz und ein VRML-Konverter als Beispielprogramm enthalten.
Die Klassen- und Methodenbeschreibungen sind knapp gehalten, manchmal 
zu knapp. So fehlt zu einigen Methoden die Beschreibung g"anzlich, f"ur 
andere ist sie wenig aussagekr"aftig. Dennoch ist die Dokumentation wie auch 
das Beispielprogramm sehr hilfreich. Manchmal kann auch ein Blick in 
den gut strukturierten Code Klarheit bringen. Man arbeitet sich relativ 
schnell in die Bibliothek ein. Alle notwendigen Geometriemodelle k"onnen 
mit dime extrahiert werden. Dime wird seit 1999 entwickelt und weiterhin 
gepflegt. Die aktuelle Version enth"alt den Stand vom Oktober 2002. Dime 
ist als Importbibliothek f"ur die beschrieben Zwecke geeignet.

Als Alternative zu dime wird seit August 2000 \textbf{dxflib} entwickelt.
Die dxflib-Projektseite findet sich auf 
SourceForge\footnote{\url{http://www.sorceforge.net}}, eine popul"are Seite 
f"ur OpenSource-Projekte. Die getestete Version ist von April 2002, dxflib 
wird auch weiterhin gepflegt. Einige Vermerke zu \emph{libdxf} scheinen 
eigentlich dxflib gewidmet. So beziehen sich die Eintr"age im 
Debian-Bugreport unter libdxf offensichtlich auf dxflib. 
Die Installation erfolgt analog zu der von dime. Es m"ussen jedoch 
noch zus"atzliche Bibliotheken installiert werden, die nicht im Standardumfang 
der Distribution enthalten sind. F"ur dxflib existiert eine Klassenreferenz.
Dxflib scheint sich noch in einem fr"uhen Stadium zu befinden. 
Wesentliche Funktionen f"ur den notwendigen Import 
sind noch nicht enthalten. Dxflib ist somit in der aktuellen Version als 
Importbibliothek unbrauchbar. Die laufende Pflege verspricht aber f"ur 
zuk"unftige Zeiten Besserung.

Zu \textbf{libdxf} finden sich nur eine begrenzte Anzahl von Verweisen. 
Libdxf wurde seit 1997 entwickelt, wird aber anscheinend seit 1999 nicht
mehr weitergepflegt. Der tar-Ball\footnote{So werden in der Unix-Welt h"aufig 
\texttt{tar.gz}-Archive genannt.} l"asst sich "ahnlich zu dime installieren 
und enth"alt neben dem Source-Code ein Beispielprogramm. Mit einigen der 
Beispielgeometrien konnte das Beispielprogramm von libdxf nicht umgehen. 
Eine weitere Analyse zeigte, dass dies nicht am Beispielprogramm sondern 
an der mangelnden Unterst"utzung durch die Bibliothek liegt. Libdxf ist somit 
als Importbibliothek unbrauchbar.

Als Importbibliothek wurde deshalb dime verwendet wird.

\section{Export}
\label{export_pot}
\alg{alg_writepot}{\procdef{writePot}{depth, filename}}{%
\pre{\begin{itemize}
     \item\param{filename} enth"alt den Namen einer beschreibbaren Datei.
     \end{itemize}}\\
\procdef{writePot}{Depth depth, String filename}
\procbeg
  \proc{out.open}{filename}\\
  \proc{out.writeLine}{depth}\\
  wCount:= 0\\
  wBinary:= 0\\
  \proc{writeTree}{\func{getTree}{}}\\
  \ifthen{wCount > 0}
    wBinary:= \(2^{8-\mbox{wCount}} *\) wBinary\\
    \proc{out.writeChar}{wBinary}
  \closeif
  \proc{out.close}{}
\closeproc
}

Zur permanenten Speicherung der Oktalbaumstruktur wird sie in einem 
pr"aorder-traversierten Bin"arstrom geschrieben, der der Ausgabedatei 
zugeordnet ist (vgl. \algref{alg_writepot}). 
Das hier als POT-Format bezeichnete Format besitzt folgende Struktur:  

\alg[!t]{alg_oct_to_potstream}{\procdef{writeTree}{tree}}{%
\pre{\begin{itemize}
     \item\param{out} ist der der Ausgabedatei zugeordnete ge"offnete 
        Ausgabestrom.
     \item\param{wCount} enth"alt die Anzahl der 'gecachten' Bits
	-- also beim externen Aufruf von \proc{writeTree}{} $0$.
     \item\param{wBinary} enth"alt die 'gecachten' Bits
	-- also beim externen Aufruf von \proc{writeTree}{} $0$.
     \end{itemize}}\\ 
\procdef{writeTree}{\_octree tree}
\procbeg
  Node node:= \func{getNode}{tree}\\
~\\
  \ifthen{\func{isLeaf}{node}}
    \integer~colBit:= 0\\
    \ifthen{\func{getColor}{node} \not= \const{NO\_OBJECT\_COLOR}}
      colBit:= 1
    \closeif
    wBinary:= 4*wBinary + colBit\\
    wCount:= wCount + 2
  \ifelse
    wBinary:= 2*Binary + 1\\
    wCount:= wCount + 1
  \closeif
~\\
  \ifthen{wCount \ge 8}
    wCount:= wCount / 8\\
    \integer~bitsToWrite:= wBinary / \(2^{\mbox{wCount}}\)\\
    \proc{out.writeChar}{bitsToWrite}\\
    wBinary:= wBinray - bitsToWrite
  \closeif
~\\
  \ifthen{\neg \func{isLeaf}{node}}
    \forloop{PartType~i}{0}{2^{\dim}-1}{1}
      \proc{writeTree}{\func{getChild}{tree, i}}
    \closefor
  \closeif
\closeproc
}

Als Dateikopf (Header) dient eine Zeile im ASCII-Format, die die maximale 
Baumtiefe enth"alt.
Es folgt der in Pr"aorder-Traversierung (vgl. Abschnitt 
\emph{Pr"aorder-Traversierung} auf Seite \pageref{pot_code}) 
umgewandelte Oktalbaum, wof"ur \algref{alg_oct_to_potstream} verwendet 
werden kann. 

F"ur einen inneren Knoten wird dabei eine \texttt{1} geschrieben, 
ein Blattknoten der Farbe \const{NO\_OBJECT\_COLOR} wird durch \texttt{00} 
repr"asentiert und alle anderen Knoten durch \texttt{01}. 

Um eine m"oglichst kompakte Darstellung zu erhalten, werden jeweils acht 
Bits des Bin"arstroms zu einem Byte zusammengefasst. 
(Es wird also \emph{nicht} f"ur einen Knoten ein Byte verwendet.)

%% End of Document