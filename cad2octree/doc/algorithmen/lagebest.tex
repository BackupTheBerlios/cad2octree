% algorithmen/lagebest.tex
% Lagebstimmung geometrischer Objekte im Raum
%
% Ausarbeitung zur Diplomarbeit Nr. 2035 - "Erzeugung und Evaluierung 
% von Oktalbaumstrukturen als Schnittstelle zu CAD-Programmen"
%
% Autor: Stefan Mahler 2002/2003
%   Universitaet Stuttgart, SgS
% Betreuer: Ralf Mundani

\subsection{Lagebestimmung geometrischer Objekte}
\label{lagebest}
Die Bestimmung der Lage eines Objekts bez"uglich eines anderen hat zur 
Oktalbaumgenerierung eine herausragende Bedeutung. Schlie"slich wird 
die Geometrie des Oktalbaums "uber die Frage '\emph{Befindet sich das Objekt 
ganz innerhalb oder au"serhalb oder teilweise inner-/au"serhalb einer 
bestimmten Zelle?}' definiert. 

\subsubsection{Punkt-in-Zellen-Test}
Um einen Punkt einer Zelle zuordnen zu k"onnen, m"ussen die realen Koordinaten 
des Punkts im gegebenen CAD-Modell auf die Koordinaten des virtuellen Gitters 
abgebildet werden. Die Abbildung der CAD-Punkte auf das Gitter soll als 
\type{GeomPoint} bezeichnet werden. 

\begin{description}
\item[Zuordnung eines realen Punktes zu einer Zelle unterster Baumebene]~\\
\label{point_in_cell_lowest_level}
Zun"achst wird von Zellen der tiefsten Ebene ausgegangen. Sie besitzen die 
Knotenh"ohe $h_{\mbox{idx}}=0$ bzw. die Tiefe $d_{t_{\max}}$. 
In jeder Achsrichtung befinden sich $2^{d_{t_{\max}}}$ Zellen, die mit $0$ 
beginnend durchnummeriert seien. 
F"ur einen beliebigen Zellindex $\mbox{\bf idx}$ dieser Ebene gilt deshalb 
\gl{gl_idx_in_octree_basenode}{\mbox{\bf idx} \in \left[ \left( 
	\left(0\right)^{\dim} | ~ 0 \right) ; \left( 
	\left(2^{d_{t_{\max}}} - 1\right)^{\dim} | ~ 0 \right) 
    \right]\mbox{.}} 
Nachbarzellen besitzen in jede Achsrichtung den Index-Abstand $1$. F"ur ein 
\type{GeomPoint} ${\bf g}$ gilt: 
\gl{gl_geompoint_idx_basenode}{{ \bf g} \in \mbox{\bf idx} : 
    { \bf g}[i] \in \left[\mbox{\bf idx}[i]; 
	\mbox{\bf idx}[i] + 1 \right) ~ 
    \land ~ h_{ \bf g} = h_{\mbox{\bf idx}} = 0%\mbox{,}~ 
%    ~\forall i \in [0; \dim)\mbox{.}
}
mit $i = 0,\dots, {\dim}-1$. Dabei sind ${\bf g}[i]$ und $\mbox{\bf idx}[i]$ 
die $i$-te Koordinate des \type{GeomPoint} bzw. der Anteil in $i$-Richtung des 
Index. Alle Zellen besitzen die gleiche Breite. Unter Verwendung 
\glref{gl_knotentiefe_zellgroesse} ergibt sich eine Maschenweite $\lambda$  
von \[\lambda = \max_{i \in \left[0; {\dim}\right)} { \left\{ 
    { \bf p}_{\max}[i] - { \bf p}_{\min}[i] \right\} } ~ / 
    ~ 2^{d_{t_{\max}}}\mbox{.}\] 
Dabei enthalten ${\bf p}_{\max}$ und ${\bf p}_{\min}$ die jeweils gr"o"sten 
bzw. kleinsten Koordinaten der boundary cube. Somit kann jeder Punkt 
${\bf p}$ des CAD-Modells mit ${\bf p} \in \left[{\bf p}_{\min}; 
{\bf p}_{\max}\right)$ mit Hilfe der linearen Abbildung 
\gl{gl_p_to_g_using_lambda}{{\bf g} = ({ \bf p} - { \bf p}_{\min}) 
    ~ / ~ \lambda}
als entsprechender \type{GeomPoint} ${\bf g}$ dargestellt werden. 
Wird ${\bf p}_{\min}$ durch 
\gl{gl_p_min}{{ \bf p}_{\min} = \left( \min_{{ \bf p}} 
    \left\{ { \bf p}[i] \right\} \right)^{i = 0,\dots,\dim - 1}}
bzw. ${\bf p}_{\max}$ durch 
\gl{gl_p_max}{{ \bf p}_{\max} = \left( \max_{{ \bf p}} 
    \left\{ { \bf p}[i] \right\} \right)^{i = 0,\dots,\dim - 1}} 
definert, k"onnen jedoch die Punkte nicht in den Oktalbaum abgebildet werden, 
die mindestens eine ${\max}$-Koordinate enthalten. Das sind alle Punkte der 
drei Randfl"achen der boundary cube $x = { \bf p}_{\max}[0]$, $y = 
{ \bf p}_{\max}[1]$ bzw. $z = { \bf p}_{\max}[2]$. Solche Punkte w"urden einer 
Zelle zugeordnet, die wegen \glref{gl_geompoint_idx} einen Index-Anteil 
von $2^{d_{t_{\max}}}$ ergeben und mit \glref{gl_idx_in_octree} au"serhalb 
des Oktalbaums liegen. Eine L"osung w"are, ${\bf p}_{\max}$ in jeweils jede 
Achsrichtung um $\epsilon : \epsilon > 0$ nach 'rechts' zu verschieben, also 
${\bf p}_{\max}$ anstatt mit Hilfe von \glref{gl_p_max} durch 
\[{ \bf p}_{\max} = \left( \max_{{ \bf p}} 
    \left\{ { \bf p}[i] \right\} + \epsilon \right)^{i = 0,\dots,\dim - 1}\] 
zu definieren. Der "ubersch"ussige relative Anteil w"are dann jedoch vom 
CAD-Modell abh"angig. Besitzt die Geometrie nur kleine Abmessung, w"are dieser 
Anteil unn"otig gro"s. Alternativ wird die Maschenweite auf 
\gl{gl_lambda_basenode}{\lambda = { \max_{i \in \left[0; {\dim}\right)} 
    { \left\{ {\bf p}_{\max}[i] - { \bf p}_{\min}[i] \right\} } ~ 
    \over ~{ 2^{d_{t_{\max}}} - \epsilon } } }
vergr"o"sert. Um eine besonders kleine Maschenweite zu erhalten, wird 
$\epsilon \ll 1$ gesetzt. 

Der \type{GemPoint} ${\bf g}$ befindet 
(vgl. \glref{gl_geompoint_idx_basenode}) sich in der Zelle mit dem 
Index $\mbox{\bf idx}$ 
\gl{gl_gp_to_idx}{\mbox{\bf idx} = \trunc{{ \bf g}}\mbox{.}}

\item[Verallgemeinerung auf beliebige Zellebenen]\label{point_in_cell_allg}~\\
Die im letzten Punkt gemachten Betrachtungen lassen sich auf beliebige 
Zell\-ebenen "ubertragen. Es gelten dementsprechend f"ur beliebige H"ohen des 
die Zelle repr"asentierenden Oktalbaumknotens $h$ (bzw. Tiefen $d$) analog zu 
\glref{gl_idx_in_octree_basenode}, \glref{gl_geompoint_idx_basenode} und 
\glref{gl_lambda_basenode}:
\gl{gl_idx_in_octree}{\mbox{\bf idx} \in \left[ \left( \left(0\right)^{\dim} 
	| ~ h \right) ; \left( \left(2^d - 1\right)^{\dim} | ~ h \right) 
    \right]\mbox{,}} 
\gl{gl_geompoint_idx}{{\bf g} \in \mbox{\bf idx} : 
    { \bf g}[i] \in \left[\mbox{\bf idx}[i]; 
	\mbox{\bf idx}[i] + 1 \right) ~ 
    \land ~ h_{ \bf g} = h_{\mbox{\bf idx}}}
mit $i = 0, \dots ,\dim-1$ bzw.
\gl{gl_lambda}{\lambda = { \max_{i \in \left[0; {\dim}\right)} 
    { \left\{ {\bf p}_{\max}[i] - { \bf p}_{\min}[i] \right\} } ~ 
    \over ~{ 2^d - \epsilon } } }
und somit
\gl{gl_p_to_g_allg}{{\bf g} = \left( ({ \bf p} - { \bf p}_{\min}) 
    ~ * ~ \frac{2^d - \epsilon }{ \max_{i \in [0,\dots,\dim-1)} \left\{ 
	{ \bf p}_{\max}[i] - { \bf p}_{\min}[i] \right\} } 
    \left|\right. h \right)\mbox{.}}

\item[Zuordnung des Zellindex zu einem \type{GeomPoint}]~\\
$g_{\mbox{\bf idx}_{\downarrow}} = \mbox{\bf idx}$ liefert den 'rechten 
unteren vorderen' \type{GeomPoint} der Zelle $\mbox{\bf idx}$.  
Als \type{GeomPoint} $g_{\mbox{\bf idx}}$ einer Zelle wird h"aufig ihr 
Mittelpunkt definiert. Wir setzen 
\gl{gl_idx_to_geompoint}{g_{\mbox{ \bf idx}} = \mbox{ \bf idx} 
    + \left( \onehalf \right)^{\dim}\mbox{.}}

\item[Punktgleichheit]~\\
Wann sind zwei Punkte ${\bf g}_1$ und ${\bf g}_2$ im Oktalbaum als identisch 
anzusehen? Zum einem k"onnte man zwei Punkte als gleich betrachten, wenn sie 
zur gleichen Zelle geh"oren, also $\trunc{{\bf g}_1} = \trunc{{\bf g}_2}$. 
Unter anderem f"ur den \vpageref{point_in_plane_test} behandelten 
\emph{Punkt-in-Ebene-Test} eignet sich eine Abstandsbeziehung besser. 
Die Eckpunkte einer Zelle ${\bf g}_{{\mbox{ \bf idx}^{\ast}}}$ haben vom 
Mittelpunkt ${\bf g}_{\mbox{\bf idx}}$ einen Abstand $a = \sqrt{3 * 
{\onehalf}^2}$. Wir definieren 
\gl{gl_geompoint_equ}{ { { \bf g}_1 = { \bf g}_2 } ~ \equival ~
    { \left| { \bf g}_1 - { \bf g}_2 \right|^2 < \threequarters }\mbox{.}}
\end{description}

\subsubsection{Punkt-auf-Strecke-Test}
Ein Punkt ${\bf Q}$ befindet sich auf der Strecke $\overline{{\bf A}{\bf B}}$
falls
\begin{itemize}
\item ${\bf Q}$ ein Endpunkt der Strecke ist, also ${\bf Q} = {\bf A}$ oder 
    ${\bf Q} = {\bf B}$, oder 
\item ${\bf Q}$ sich auf der Geraden, die durch ${\bf A}$ und 
    ${\bf B}$ verl"auft, befindet und zwischen ${\bf A}$ und ${\bf B}$ liegt.
\end{itemize}

${\bf Q}$ liegt auf der Geraden ${\bf A}{\bf B}$, wenn der Fu"spunkt 
${\bf F}_{\bf Q}$ zu ${\bf Q}$ auf der Geraden ${\bf A}{\bf B}$ mit ${\bf Q}$ 
zusammenf"allt, also gilt
\gl{gl_q_at_line}{{ \bf Q} = { \bf F}_{ \bf Q}\mbox{.}}
Der Fu"spunkt kann mit Hilfe des Skalarprodukts berechnet werden:
\gl{gl_footpoint_at_line}{ \vct{O{F_Q}} = \frac{ < \vct{AQ}; \vct{AB} > }{%
	| \vct{AB} |^2 } ~ \vct{AB} \: + \: \vct{OA}\mbox{.}}

Ein Punkt ${\bf Q}$, der sich auf der Geraden ${\bf A}{\bf B}$ befindet, liegt 
zwischen ${\bf A}$ und ${\bf B}$, wenn gilt: 
\gl{gl_footpoint_at_stretch}{ | \overline{AQ} | < | \overline{AB} | ~ 
    \land ~ | \overline{QB} | < | \overline{AB} |\mbox{.}}

\subsubsection{Punkt-in-Ebene-Test}
\label{point_in_plane_test}
Im Folgenden werden zwei Methoden erl"autert, um zu bestimmen, ob sich ein 
Punkt ${\bf Q}$ in der Ebene $E$, die durch die Punkte ${\bf A}$, 
${\bf B}$ und ${\bf C}$ gegeben ist, befindet.
Dabei ist zu beachten
\begin{enumerate}
\item\textbf{Sonderf�lle:} Sind ${\bf A}$, ${\bf B}$ und ${\bf C}$ kollinear 
    (oder gar identisch), so ist undefiniert, ob ${\bf Q}$ in $E$ liegt. 
\item\textbf{\emph{'Abstandsregel'}:} Der Punkt-in-Ebene-Test wird 
    insbesondere dazu benutzt, um zu bestimmen, ob $E$ durch eine Zelle mit 
    dem Index ${\bf i}$ verl"auft. Der Zelle wird der Punkt ${\bf g}_i$ 
    zugeordnet, der den Mittelpunkt der Zelle darstellt. Verl"auft die Ebene 
    nicht durch den Zellmittelpunkt sondern lediglich durch die 
    Zellrandfl"ache, muss dennoch die Zelle als zur Ebene geh"orig angesehen 
    werden. Deshalb werden Punkte, die in einen Abstand 
    von $\le\onehalf$ zu $E$ besitzen, mit zur Ebene $E$ gerechnet.
\end{enumerate}

\begin{description}
%\renewcommand\item{\item ~\\}
\item["Uber den Fu"spunkt] \label{descr_in_plane_footpoint}~\\ 
    Gilt f"ur den Fu"spunkt ${\bf F}_{\bf Q}$ des Punktes ${\bf Q}$ auf der 
    Ebene $E$
    \gl{gl_point_in_plane_via_f}{{ \bf F}_{ \bf Q} = { \bf Q}\mbox{,}}
    dann liegt ${\bf Q}$ in $E$. 
    ${\bf F}_{\bf Q}$ kann "uber 
    \gl{gl_footpoint_in_plane}{ \vct{O{F_Q}} = \vct{OQ} ~ - ~ 
	\textstyle{ \left( \left< \vct{n_E}; \vct{OQ} \right> - d \right) } 
	\vct{n_E} }
    berechnet werden. Dabei ist $\vct{n_E}$ nach 
    \glref{gl_flaechenorientierung} der Normalenvektor der Ebene.
\item["Uber das Determinantenverfahren] \label{descr_in_plane_det}~\\ 
    Das Volumen $V$ des durch die Vektoren $\vct{a}$, $\vct{b}$ und $\vct{c}$ 
    gegebenen Spans kann durch 
    \gl{gl_vol_ueber_det}{V = \det \left(\vct{a}; \vct{b}; \vct{c}\right)}
    ermittelt werden. 
    Damit ergibt sich folgende Beziehung f"ur den Abstand $a$ des 
    Punktes ${\bf Q}$ zur Ebene $E$ 
    \gl{gl_dist_point_plane}{\det{}^2 \left(\vct{AB}; \vct{AC}; \vct{AP}\right) 
	< a^2 ~ \left| \vct{AB} \crossprod \vct{AC} \right|^2\mbox{.}} 
    Mit einem maximal zul"assigen Abstandsquadrat von $a^2 = \onehalf$ ergibt 
    sich
    \gl{gl_point_in_plane_via_det}{{ \bf Q} \in E ~ \Longleftrightarrow ~ 
	\det{}^2 \left(\vct{AB}; \vct{AC}; \vct{AP}\right) 
	\le \onehalf ~ \left| \vct{AB} \crossprod \vct{AC} \right|^2\mbox{.}} 
\end{description}

\subsubsection{Punkt-in-Polygon-Test}
F"ur den Test, ob sich ein Punkt ${\bf Q}$ in einem \emph{beliebigen} Polygon 
befindet, bietet sich die \textbf{Strahlmethode} an:

Sei ein Polygon $H$ durch den geschlossenen Streckenzug 
$\overline{{\bf p}_0{\bf p}_1}$ $\dots$ 
$\overline{{\bf p}_{n-2}{\bf p}_{n-1}}$ $\overline{{\bf p}_{n-1}{\bf p}_n}$ 
mit ${\bf p}_n = {\bf p}_0$ gegeben. Dann kann mit Hilfe des Teststrahls 
$\strch{{\bf Q}{\bf W}}$ "uberpr"uft werden, ob ${\bf Q}$ auf dem Rand, 
innerhalb oder au"serhalb von $H$ liegt. Dabei muss garantiert sein, dass 
sich ${\bf W}$ au"serhalb von $H$ befindet. Zu diesem Zweck wird ${\bf W}$ 
am besten auf eine Position au"serhalb der boundary cube festgelegt, z.B. : 
${\bf W} = \left({\bf Q}[0]; {\bf Q}[1]; 2^{d_{\bf Q}} | h_{\bf Q}\right)$.
Nun wird f"ur alle Streckenz"uge $\overline{{\bf p}_{i}{\bf p}_{i+1}}$ mit 
$i= 0,\dots,n-1$ "uberpr"uft, ob sie den Teststrahl schneiden. 
Liegt ${\bf Q}$ auf einem Streckenzug, liegt ${\bf Q}$ auf dem Polygonrand 
und der Test kann abgebrochen werden. Ist ansonsten die Anzahl der Schnitte 
gerade, dann liegt ${\bf Q}$ innerhalb, ist die Anzahl der Schnitte ungerade, 
befindet sich ${\bf Q}$ au"serhalb des Polygons, also gilt: 
\gl{gl_point_in_polygon_via_ray}{{ \bf Q} \cases{
    \mbox{on} ~ H\mbox{,} & { \bf Q} \in \overline{{ \bf p}_i{ \bf p}_{i+1}}\\
    \mbox{in} ~ H\mbox{,} & \nu \mod 2 = 0\\
    \notin    ~ H\mbox{,} & \nu \mod 2 = 1}\mbox{.}}
Dabei ist $i= 0,\dots,n-1$ und $\nu$ die Anzahl der Schnitte. Folgende 
Sonderf"alle sind zu beachten: Ein Streckenzug wird in einem Eckpunkt vom 
Teststrahl geschnitten oder f"allt mit dem Teststrahl zusammen. Diese F"alle 
werden genau dann als Schnitt gez"ahlt, wenn es im Folgenden zu einem 
Seitenwechsel bez"uglich des Teststrahls kommt.

\label{polyeder_test}
Das Verfahren l"asst sich auch auf beliebige \textbf{Polyeder} (dann mit Test 
auf Schnitt der Teil\-oberfl"achen) "ubertragen. 

Alternativ l"asst sich das Verfahren leicht abwandeln, wenn man die 
\emph{Orientierung} der Polygon-/Polyeder-R"ander ausnutzt 
(\textbf{Teststrahlmethode}). Jetzt muss nur der zu ${\bf Q}$ am n"achsten 
gelegene Streckenzug bzw. die zu ${\bf Q}$ am n"achstengelegene 
Teiloberfl"ache\footnote{wieder entlang eines Teststrahls oder alternativ 
der/die allgemein am n"achsten gelegene Streckenzug bzw. Teil\-oberfl"ache} 
ermittelt werden. Dann kann mit Hilfe des unten 
beschriebenen Tests der \emph{Lage eines Punktes bez"uglich einer Ebene} 
(als Basispunkt der Ebene den Schnittpunkt ${\bf S}$ einsetzen) bestimmt 
werden, ob sich der Punkt innerhalb oder au"serhalb des Polyeders befindet. 
Liegt ${\bf Q}$ in der Ebene der Teiloberfl"ache, so wird diese Fl"ache 
einfach unber"ucksichtigt gelassen. Der Sonderfall ${\bf Q}$ liegt in der 
Teilfl"ache wurde zuvor durch ${\bf Q} = {\bf S}$ "uberpr"uft. Befindet 
sich ${\bf Q}$ in der Ebene, aber nicht auf der Teilfl"ache, muss es 
mindestens eine weitere Fl"ache geben, die genauso weit von ${\bf Q}$ entfernt 
ist, deren Ebene jedoch nicht mit dem Teststrahl $\strch{{\bf Q}{\bf S}}$ 
zusammenf"allt. Kann keine n"achstliegende Fl"ache gefunden werden, befindet 
sich ${\bf Q}$ au"serhalb des Polyeders. 
Prinzipiell l"asst sich das Verfahren auch auf Objekte mit gekr"ummten Rand 
"ubertragen. Dabei ist jedoch -- neben der schwereren Schnittpunktermittlung 
-- zu beachten, dass mehrere Schnittpunkte (nicht nur im Sonderfall des 
Zusammenfallens von Streckenzug/Fl"achenst"uck mit dem Teststrahl) 
existieren k"onnen. Des Weiteren ist die Orientierung gekr"ummter Fl"achen 
i.A. nicht global gleich und muss dehalb am entsprechenden Schnittpunkt 
ermittelt werden. 

Werden nicht beliebige, sondern nur \emph{konvexe Polygone} eingesetzt, 
k"onnen effizientere oder einfachere Verfahren gefunden werden: 

Besonders effizient ist dabei das \textbf{Umlaufsinnverfahren}. 
Sei ein Dreieck $H= \triangle{\bf A}{\bf B}{\bf C}$ im 
\emph{Zweidimensionalen} gegeben. 
Der Punkt ${\bf Q}$ befindet sich genau dann in $H$, wenn der Umlaufsinn der 
Dreiecke die durch Ersetzen von jeweils einem der Punkte ${\bf A}$, ${\bf B}$ 
bzw. ${\bf C}$ durch ${\bf Q}$ entstehen, mit dem Umlaufsinn von $H$ identisch 
ist, also 
\gl{gl_point_in_triangle_ccw}{\func{ccw}{{\large H}} = 
    \func{ccw}{$\triangle${\bf Q}{\bf B}{\bf C}} = 
    \func{ccw}{$\triangle${\bf A}{\bf Q}{\bf C}} = 
    \func{ccw}{$\triangle${\bf A}{\bf B}{\bf Q}}\mbox{.}}
Dabei ist \func{ccw}{} ein Funktion, die zu einem Dreieck seinen 
Umlaufsinn bestimmt. Die Umlaufsinnbestimmung kann im zweidimensionalen 
Fall mit Hilfe des Skalarprodukts sehr effizient geschehen (vgl. 
\cite{algo_cpp}). Die "Ubertragung ins Dreidimensionale erweist sich als 
schwieriger. Hier gibt es keine nat"urliche Definition des Uhrzeigersinns. 
Als L"osung k"onnten im Falle, dass ${\bf Q}$ in der Ebene des Dreiecks $H$ 
liegt, die betreffenden Punkte in einer (zweidimensionalen) Basis 
zweier Ebenenvektoren dargestellt werden. Alternativ k"onnten die Richtungen 
der Normalvektoren der Dreiecke verglichen werden.

\label{point_in_polygon_angle}
\cite{point_in_polygon} stellt das in dieser Arbeit verwendete 
\textbf{Winkelverfahren} vor. Ein Punkt ${\bf Q}$, der in der Ebene $E$ des 
konvexen Polygons $H$ liegt, befindet sich genau dann innerhalb von $H$, 
wenn gilt:
\gl{gl_point_in_polygon_angle}{{\sum_{i=0}^{n-1} \arccos{
    \left<{\vct{Q{p_i}} \over |\vct{Q{p_i}}|}; 
	  {\vct{Q{p_{i+1}}} \over |\vct{Q{p_{i+1}}}|} \right>}} = 2\pi\mbox{.}}
Dabei ist wieder $\overline{{\bf p}_0{\bf p}_1}, \dots, 
\overline{{\bf p}_{n-1}{\bf p}_n}$ (mit ${\bf p}_n={\bf p}_0$ und 
${\bf Q} \notin \overline{{{\bf p}_i}{{\bf p}_{i+1}}}$) der den Rand 
von $H$ definierende geschlossen Streckenzug. 

F"ur konvexe Polygone kann auch das \textbf{Eckpunkt-Schnittverfahren} 
verwendet werden. 
Ein Punkt ${\bf Q}$ der sich nicht auf dem Polygonrand befindet, liegt genau 
dann im Polygon $H$, welches durch den Streckenzug 
$\overline{{\bf p}_0{\bf p}_1}$,\dots,$\overline{{\bf p}_{n-1}{\bf p}_n}$ 
mit ${\bf p}_n={\bf p}_0$ gegeben ist, wenn gilt
\gl{gl_point_in_polygon_corner_intersect}{%
    \forall i \in [0; n) \forall k \in [0; n) \setminus \{i; i+1\}:
    \overline{{\bf Q}{\bf p}_i} \land \overline{{\bf p}_k{ \bf p}_{k+1}}
    = \emptyset\mbox{,}} 
also keine Strecke $\overline{{\bf Q}{\bf p}_i}$ durch einen Streckenzug des 
Polygons $\overline{{\bf p}_k{\bf p}_{k+1}}$ geschnitten wird.

\subsubsection{Lage eines Punktes bez"uglich einer Fl"ache}
Aus der Orientierung der Fl"achen nach \glref{gl_flaechenorientierung} ergibt 
sich f"ur alle Fl"achen eine Innen- und Au"senseite. 
F"ur einen Punkt ${\bf Q}$, der sich nicht auf der Ebene befindet, kann nun 
durch 
\gl{gl_point_plane_side}{{\bf Q} ~ \mathrm{is} ~ 
        \cases{\mbox{inside},  & \varphi > 0\\
	       \mbox{outside}, & \varphi < 0}}
sein Lage berechnet werden. $\varphi$ ist ein dem Winkel 
zwischen dem Normalenvektor $\vct{n_E}$ und dem Vektor zwischen ${\bf Q}$ 
und einem Fl"achenpunkt ${\bf F}$ entsprechender Wert. Es gilt 
\gl{gl_phi_from_skalar_prod}{\varphi = \left<\vct{n_E}; \vct{QF}
	\right>\mbox{.}}

%% End of Document
