% Ausarbeitung zur Diplomarbeit Nr. 2035 - "Erzeugung und Evaluierung 
% von Oktalbaumstrukturen als Schnittstelle zu CAD-Programmen"
%   
% Autor: Stefan Mahler 2002/2003
%   Universitaet Stuttgart, SgS
% Betreuer: Ralf Mundani

\section{Spline-Fl"achen-Generierung}
\label{algo_spline}
In den letzten Abschnitten wurden M"oglichkeiten der Oktalbaumgenerierung 
f"ur glatte K"orperoberfl"achen ausf"uhrlich beschrieben. Zur 
Oktalbaumerzeugung mit Spline-Oberfl"achen wird ein Trick benutzt: 

Die Genauigkeit des Oktalbaummodells ist durch das durch die Zellen der 
untersten Ebene entstehende Gitter begrenzt. Durch \glref{gl_geompoint_equ} 
ist gegeben, wann zwei Punkte als gleich angesehen werden k"onnen. Eine 
Oberfl"ache, die durch entsprechend kleine Polygone approximiert wurde, wird 
demnach durch den selben Oktalbaum repr"asentiert, wie die urspr"ungliche 
Spline-Oberfl"ache. Zur Oktalbaumgenerierung wird die Spline-Oberfl"ache 
demnach durch ein Polygonnetz ersetzt, welches die Spline-Fl"ache hinreichend 
genau ann"ahert. 

\alg{alg_getfacepoint}{\funcdef{getFacePoint}{u, v}{GeomPoint}}{%
\pre{\(u \in [0; \max\{u^{\ast}_i | u^{\ast}_i \in U\}) \land 
       v \in [0; \max\{v^{\ast}_i | v^{\ast}_i \in V\}) \)}\\
\funcdef[\\\hspace{1em}]{getFacePoint}{Coordinate u, Coordinate v}{GeomPoint}
\funcbeg
  \field{0..3}{Coordinate} N\_u\\
  \field{0..3}{Coordinate} N\_v\\
~\\
  \integer~u\_span:= \func{findSpan}{\const{U\_DIR}, u}\\
  \proc{basisFuns}{\const{U\_DIR}, u\_span, u, N\_u}\\
  \integer~v\_span:= \func{findSpan}{\const{V\_DIR}, v}\\
  \proc{basisFuns}{\const{V\_DIR}, v\_span, v, N\_v}\\
~\\
  GeomPoint x:= \( (0; 0; 0| 0)\)\\
  \integer~u\_ind:= u\_span - 3\\
  \forloop{\integer~k}{0}{3}{1}
    GeomPoint temp:= \( (0; 0; 0| 0)\)\\
    \integer~v\_ind:= v\_span - 3\\
    \forloop{\integer~l}{0}{3}{1}
      temp:= temp \(+~N_u[l] *
                      { \bf c}_{u_{\mathrm{ind}}+l; v_{\mathrm{ind}}+k}\)\\
    \closefor
    x:= x + N\_v[k] * temp\\
  \closefor
  \ret{x}
\closefunc
}

\alg{alg_findspan}{\funcdef{findSpan}{extent, u}{\integer}}{%
\funcdef{findSpan}{Extent extent, Coordinate u}{\integer}
\funcbeg
  \ifthen{\func{isClosed}{extent}}
    \ret{\(\trunc{u}\)}
  \closeif
~\\  
  n:= \func{getControllpointCount}{extent}\\
  \ifthen{u \ge n}
     \ret{n}
  \closeif
~\\  
  \ret{\(\trunc{u} + 3\)}
\closefunc
}

\alg{alg_basisfuns}{\procdef{basisFuns}{extent, i, u, \outparam{N[]}}}{%
\pre{\(u \in [\func{getKnot}{extent, i}; \func{getKnot}{extent, i+1})\)}\\
\procdef{basisFuns}{Extent extent, \integer~i, Coordinate u, \\
\hspace{11em}\outparam{\field{0..3}{Coordinate} N}}
\procbeg
  \field{0..3}{Coordinate} left\\
  \field{0..3}{Coordinate} right\\
~\\
  N[0]:= 1\\
  \forloop{\integer~k}{1}{3}{1}
    left[k]:= u - \func{getKnot}{extent, i + 1 - k}\\
    right[k]:= \func{getKnot}{extent, i + k} - u\\
    Coordinate saved:= 0\\
    \forloop{\integer~r}{0}{k-1}{1}
      Coordinate temp:= N[r] / (left[k-r] + right[r+1])\\
      N[r]:= saved + temp*right[r+1]\\
      saved:= left[k-r] * temp\\
    \closefor
    N[k]:= saved\\
  \closefor
\closeproc
}

Um f"ur ein Parameterwertpaar $(u; v)$ den entsprechenden 
Oberfl"achenpunkt ${\bf x}(u; v)$ zu bestimmen, werden die sehr effizienten 
Algorithmen \ref{alg_getfacepoint}, \ref{alg_findspan} und \ref{alg_basisfuns} 
verwendet, die auf den in \cite{nurbs_book} beschriebenen Methoden 
\emph{SurfacePoint()}, \emph{FindSpan()} und \emph{BasisFuns()} basieren. 

\param{extent} gibt an, welche der beiden Richtungen der Splinefl"ache 
betrachtet werden soll. Als Knotenvektoren dienen $U$ und $V$. 
Die zugeh"origen (von $0$ verschiedenen) Basisfunktionen werden in $N_u$ und 
$N_v$ gespeichert. 
${\bf c}_{l;k}$ bezeichnet den $(l;k)$-ten Kontrollpunkt der Spline.

Man beachte, dass die in \cite{nurbs_book} beschriebenen und in den 
Algorithmen \ref{alg_getfacepoint}, \ref{alg_findspan} und 
\ref{alg_basisfuns} verwendeten Knoten $u^\ast_i$ offener B-Splines aus 
implementierungstechnischen Gr"unden einen Indexwert $i \in [0; n+p+1)$ haben. 
Im Gegensatz dazu hatten die im Abschnitt \ref{spline} eingef"uhrten 
Spline-Knoten $u_{i\alt}$ Indexwerte $i{\alt} \in [-p-1;n)$. Dabei 
ist $n$ die Anzahl der Kontrollpunkte und $p$ der Grad der Spline-Kurve.

Da die in der Abteilung erstellten DXF-Modelle ausschlie"slich kubische 
B-Splines als Spline-Fl"achen enthalten, wurde das Verfahren auf diese 
Gruppe von Splines eingeschr"ankt. (Eine Erweiterung auf beliebige Splines 
ist jedoch sehr einfach m"oglich.) 
Der Knotenvektor $U = \left\{u^\ast_i\right\}$ einer 
Ausdehnungsrichtung "uber $n$ Kontrollpunkte ist f"ur eine offene Spline mit 
\gl{gl_knotvector_open}{u^\ast_i = 
	\cases{0,   & i \le 3\\
	       i-3, & i \le n\\
	       n-3, & i  >  n}}
($i \in [0; n+p+1)$) und im geschlossenen Fall mit
\gl{gl_knotvector_closed}{U = \left\{0, \dots, n\right\}}
gegeben.
Im geschlossen Fall seien zus"atzlich die Kontrollpunkte $c_n=c_0$, 
$c_{n+1}=c_1$,\dots und $c_{-1}=c_{n-1}$.

Durch die Polygonnetz-Approximation der B-Spline-Oberfl"ache entsteht eine 
gro"se Menge an Objekten, die auf etwaigen Schnitt mit dem Teststrahl zur 
Lagebestimmung von Zellen (vgl. Abschnitt \emph{Punkt-in-Polygon-Test} 
\vpageref{polyeder_test} und \algref{alg_locate}) durchlaufen werden m"ussen. 
F"ur die Lokalisation von Zellen innerhalb des F"ullalgorithmus (Abschnitt 
\ref{fill_alg}) ist aber unter Umst"anden die Verwendung des 
Kontrollpunktnetzes anstatt des feinen Oberfl"achenpolygonnetzes ausreichend: 

Vor Anwendung des F"ullalgorithmus ist bereits der K"orperrand in 
den Oktalbaum "ubertragen worden. Damit der F"ullalgorithmus korrekt arbeitet, 
m"ussen die zur Bestimmung der F"ullfarbe aufgew"ahlten Punkte korrekt 
lokalisiert werden. Wir nehmen an, dass das Kontrollpunktnetz nicht 
selbstschneidend ist (f"ur den modellierten K"orper muss dies ohnehin gelten 
(vgl. Abschnitt \ref{flmodels}). Des Weiteren sollen, falls ein K"orper aus 
mehreren nicht zusammenh"angenden Teilen besteht oder mehrere K"orper sich 
innerhalb eines Modells befinden, die dazugeh"origen Kontrollpunktnetze 
einander schnittfrei sein. Dann werden aufgrund der Eigenschaft, dass sich 
Spline-Oberfl"achen immer innerhalb ihres Kontrollpunktnetzes befinden, 
K"orperinnenpunkte immer richtig lokalisiert. 

Wurden keine Hohlk"orper modelliert (es gibt kein Gebiet, dass vollst"andig 
vom K"orperrand umschlossen ist und nicht zum K"orper geh"ort), muss jetzt nur 
noch der Au"senbereich richtig bestimmt werden:
Beim F"ullalgorithmus wird immer der erstm"ogliche Punkt zur Ermittlung 
der F"ullfarbe verwendet. Befindet sich die 
Zelle $\trunc{{\bf g}_{{\bf p}_{\min}}}$ nicht auf den K"orperrand, wird sie 
zur F"ullfarbenermittlung verwendet. Liegt in ihr auch kein Kontrollpunkt, 
wird der gesamte Au"senbereich korrekt mit der Farbe \const{NO\_OBJECT\_COLOR} 
markiert. Dass die Zelle $\trunc{{\bf g}_{{\bf p}_{\min}}}$ stets 'freie' 
Zelle ist (sie enth"alt kein Modellierungspunkte, wie z.B. Kontrollpunkte), 
kann allgemein durch eine entsptechende Verschiebung des 
Punkts ${\bf p}_{\min}$ nach weiter 'au"sen' (in Richtung 
$(-\infty; -\infty; -\infty)$) um eine Zelle erreicht werden (vgl. zur 
Anpassung des Oktalbaumgebiets Abschnitt \ref{point_in_cell_lowest_level}). 
Bei h"oheren Aufl"osungen m"ussen somit wesentlich weniger Polygone zur 
Lokalisation beachtet werden. Die Zahl der zu durchlaufenden Polygone ist 
--~wie bei der Modellierung glatter K"orperoberfl"achen~-- unabh"angig von 
der Oktalbaumtiefe. 

Beim klassischen Verfahren kann die Aufl"osung des Approxiamtion-Netzes f"ur 
die Spline-Fl"ache entsprechend der Knotentiefe adaptiert werden. Es muss 
also nicht schon bei der Wurzel eine Spline-Extrahierung mit Polygonen der 
Abmessungen entsprechend der Zellen der untersten Ebene erfolgen. 

%% End of Document
