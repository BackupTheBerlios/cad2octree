% algorithmen/schnitt.tex
% Schnitt zweier Linien, Schnitt Linie mit Polygon
%
% Ausarbeitung zur Diplomarbeit Nr. 2035 - "Erzeugung und Evaluierung 
% von Oktalbaumstrukturen als Schnittstelle zu CAD-Programmen"
%
% Autor: Stefan Mahler 2002/2003
%   Universitaet Stuttgart, SgS
% Betreuer: Ralf Mundani

\subsection{Schnitt geometrischer Objekte}
Im Folgenden werden einige Schnittverfahren f"ur einfache Objekte vorgestellt. 
Wie im oberen Abschnitt zu sehen ist, wird die Schnittbestimmung insbesondere 
f"ur einige Verfahren zur Lagebestimmung verwendet.

\subsubsection{Schnitt zweier Strecken}
\cite{algo_cpp} beschreibt einen besonders effizienten Algorithmus, der  
"uberpr"uft, ob sich zwei Strecken $\overline{{\bf A}{\bf B}}$ und 
$\overline{{\bf C}{\bf D}}$, die sich in einer Ebene befinden, schneiden. 
Der Einfachheit halber soll verlangt sein, dass die Punkte 
${\bf A}$, ${\bf B}$, ${\bf C}$, ${\bf D}$ paarweise verschieden 
sind und nicht kollinear liegen. 
(Der Originalalgorithmus macht diese Einschr"ankung nicht, da hier f"ur diese 
Sonderf"alle die \func{ccw}{{}${\bf p}_1$, ${\bf p}_2$, ${\bf p}_3$} speziell 
definierte Werte liefert.)
\gl{gl_intersect_two_str_ccw}{ 
    \begin{array}{c} 
    \overline{{\bf A}{\bf B}} \; \mathrm{intersect} \; 
	\overline{{\bf C}{\bf D}} \; \equival \\
    \left( \func{ccw}{{\bf A},{\bf B},{\bf C}} = 
	\func{ccw}{{\bf B},{\bf A},{\bf D}} \right) 
     ~ \land ~ \left( \func{ccw}{{\bf C},{\bf D},{\bf A}} = 
	\func{ccw}{{\bf D},{\bf C},{\bf B}} \right) 
    \end{array}
}

\subsubsection{Schnitt einer achsparallelen Geraden mit einem Polygon}
Es soll der Schnittpunkt zwischen einer achsparallelen Geraden mit einem 
Polygon $H$ bestimmt werden, wenn ein solcher existiert. 
Hier soll auf die Mittel der Vektorrechnung zur"uckgegriffen werden. 
Die Verwendung einer achsparallelen Gerade f"ur den Schnitt hat den Vorteil, 
dass bereits zwei Koordinaten des Schnittpunkts bekannt sind.  
Es sind die jeweiligen Koordinaten der Geraden. O.B.d.A. soll eine Gerade
parallel zur z-Achse betrachtet werden, die durch den Punkt 
${\bf Q} = (x_0; y_0; z_0 | h)$ verl"auft: 
\[g = \left\{ (x_0; y_0; z | h) : z \in \Realnumbers \right\}\mbox{.}\]
Dann muss f"ur den Schnittpunkt ${\bf S}=\left(x_0; y_0; z_{\bf S}\right)$ 
gelten:
\gl{gl_S_in_polygon}{\left(x_0; y_0; z_{\bf S}\right) \in H\mbox{.}}
Die Gerade liegt in der Polygonebene oder verl"auft parallel zu ihr, falls 
\gl{gl_z_line_parallel_polygon}{\vct{n_H}[2] = 0\mbox{.}}
Solche F"alle werden im Folgenden nicht weiter betrachtet.
F"ur den Schnittpunkt gilt nun 
\gl{gl_z_s_line_polygon}{z_{\bf S} = z_0 + { \scriptstyle{ \left<
    \vct{Q{p_0}}; \vct{n_H} \right>} \over \vct{n_H}[2]}
    \mbox{.}}
Zur Bestimmung, ob der Schnittpunkt mit der Polygonebene auch im Polygon 
selbst liegt, kann die Projektion in die $xy$-Ebene betrachtet und die 
Umlaufsinnmethode angewendet werden. 
 
F"ur die Strahlmethode (im Abschnitt \emph{Punkt-in-Polygon-Test} 
\vpageref{polyeder_test} beschrieben) muss 
nun noch $z_{\bf S} > {\bf Q}[2]$ sein.

%% End of Document
