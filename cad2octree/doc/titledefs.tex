% doc/titledefs.tex
% Titelplatt-Definitionen
%
% Ausarbeitung zur Diplomarbeit Nr. 2035 - "Erzeugung und Evaluierung 
% von Oktalbaumstrukturen als Schnittstelle zu CAD-Programmen"
%
% Autor: Stefan Mahler 2002
%   Universitaet Stuttgart, SgS
% Betreuer: Ralf Mundani

% Template www.informatik.uni-stuttgart.de/fakultaet/studienberatung/
%   richtlinien/diplomtitel-template.tex genutzt

\studiengang{swt}

% Die folgenden neun Angaben, ausser \art und \institut, muessen immer
% angegeben werden.  Mehrere Zeilen werden mit \\ getrennt.
%
\pruefer{Prof.~Dr.~rer.~nat.~H.-J.~Bungartz}
\betreuer{Dipl.~Inf.~R.-P.~Mundani}

\begonnen{15.~August~2002}
\beendet{14.~Februar~2003}

\titel{Erzeugung und Evaluierung von Oktalbaumstrukturen als Schnittstelle 
  zu CAD-Programmen}
\autor{Stefan~Mahler}

%\art{Studien}   %% Default ist \art{Diplom}, also eine Diplomarbeit
\nummer{2035}  % Nummer der Studien-/Diplomarbeit

\institut\ipvs %% Default ist \institut\ifi
          %% Andere Institutsnamen werden direkt mit \institut{ ... }
          %% angegeben, Zeilen werden mittels \\ getrennt.

%% Mehrere CR-Klassifikationen einfach mit \crk{..} aufzaehlen
%% (optional).
%
\crk{E.1} %Data/Data Structures (Trees)
\crk{J.6} %Computer Applications/Computer-Aided Engineering (CAD)
\crk{F.2.2} %Theory of Computation/Analysis of Algorithms and Problem 
	    %Complexity/Nonnumerical Algorithms and Problems
	    %(Geometrical problems and computations)
\crk{G.1.2} %Mathematics of Computing/Numerical Analysis/Approximation
            %(Approximation of surfaces and contours)

%% Fuer Leute mit ausgefallenem Geschmack kann der Titel/Autor-Kasten
%% eingerahmt werden. Default ist keine Umrandung.
%
%\umrandet

%%%%%%%%%%%%%%%%%%%%%%%%%%%%%%%%%%%%%%%%%%%%%%%%%%%%%%%%%