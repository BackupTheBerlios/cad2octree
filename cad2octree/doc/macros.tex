% doc/macros.tex
% Allgemeine Makro-Definitionen
%
% Ausarbeitung zur Diplomarbeit Nr. 2035 - "Erzeugung und Evaluierung 
% von Oktalbaumstrukturen als Schnittstelle zu CAD-Programmen"
%
% Autor: Stefan Mahler 2002
%   Universitaet Stuttgart, SgS
% Betreuer: Ralf Mundani

\newcommand\hide[1]{} %

\newcommand*\extpic[1]{%
    \includegraphics[width=0.4\linewidth]{#1}
}

\renewcommand\cases[1]{%
    \left\{
    \begin{array}{rl}
    #1
    \end{array}
    \right.
}

\newcommand\alg[4][t]{%
    \begin{figure}[#1]\renewcommand*\figurename{Algorithmus}
    \hrule width\textwidth\vspace{0.5em}\begin{flushleft}\begin{alltt}
    #4%\vspace{0.5em}\hrule width\textwidth
    \end{alltt}\end{flushleft}\setlength{\belowcaptionskip}{0pt}
    \setcapwidth[l]{\textwidth}\caption{#3\hspace{-0.8em}}
    \label{#2}\vspace{0.5em}\hrule width\textwidth\end{figure}}

\newcommand\bild[4]{%
    \includegraphics[width=#1]{#4}
    \begin{center}
    \renewcommand\figurename{Abb.}
    \setcapindent{1em}\captionof{figure}{#3}
    \label{#2}
    \vspace{1em}
    \end{center}
}

\newcommand\graf[3]{\begin{figure} \begin{center} \input{graf/#3} 
    \caption{#2} \label{#1} \end{center} \end{figure}
}

\newcommand*\diabeg[1][t]{\begin{figure}[#1] \begin{center}}
\newcommand*\diaend{\end{center} \end{figure}}

\newcommand\dia[7][t]{%
    \diabeg[#1] \begin{tabular}{cc}
    \includegraphics[width=0.44\linewidth]{#5}
    & \includegraphics[width=0.44\linewidth]{#7} \\
    \multicolumn{1}{c}{\emph{#4}} & \multicolumn{1}{c}{\emph{#6}} 
    \end{tabular} \caption{#3} \label{#2} \diaend}

\newcommand*\tabbeg[1][t]{\begin{table}[#1] \begin{center}}
\newcommand*\tabend[3][1]{\ifnumber{#1}{\caption{#3}}{\caption[#1]{#3}}
    \label{#2} \end{center} \end{table}}

\newcommand*\opitemlabel{$\rhd$}
\newcommand*\opitem{\item }
\newcommand\oplistbeg{\begin{itemize}\renewcommand\labelitemi{\opitemlabel}
    \renewcommand\labelitemii{\opitemlabel}
    \renewcommand\labelitemiii{\opitemlabel}}
\newcommand*\oplistend{\end{itemize}}
\newcommand\gl[2]{%
    \begin{equation} \label{#1} {#2} \end{equation}}

\newcommand*\glref[1]{(\ref{#1})}
\newcommand*\gvref[1]{(\ref{#1}) \vpageref{#1}}
\newcommand*\algref[1]{Algorithmus \ref{#1}}
\newcommand*\modelref[2]{Modell {\tt #2} Abschnitt \ref{#1}}

\newcommand*\positive{$+$\hspace{0.13em}}
\newcommand*\neutral{$\bigcirc$}
\newcommand*\negative{$-$\hspace{0.13em}}

\newcommand*\alt{^{\circ}}
\newcommand*\onehalf{\mbox{\textonehalf}}
\newcommand*\threequarters{\mbox{\textthreequarters}}

\newcommand*\Realnumbers{\mathbf{\mathcal{R}}}

\newcommand*\crossprod{ ~ \times ~ }
\newcommand*\mod{\bmod}
\newcommand*\sign{\mathrm{sign}}
\newcommand*\abs[1]{\left|#1\right|}
\newcommand*\trunc[1]{{ \left\lfloor { #1 } \right\rfloor } }
\newcommand*\equival{\iff}
\newcommand*\band{{\mathrm{and}}}

\newcommand*\vct[1]{\stackrel{\longrightarrow}{#1}}
\newcommand*\strch[1]{\stackrel{\longmapsto}{#1}}
\newcommand*{\conti}{$\mathcal{C}^1$}
\newcommand*{\contii}{$\mathcal{C}^2$}
\newcommand*\binom[2]{{{#1} \choose {#2}}}

\newcommand*\bez{B\'ezier}

\newcommand\funcdef[4][]{\texttt{\textbf{function} \textrm{#2}(\param{#3})#1
    \textbf{return}~{#4}}~}
\newcommand*\funcbeg{\\ \textbf{begin} \begin{labeling}{ii} \item}
\newcommand*\procbeg{\funcbeg}
\newcommand\procdef[2]{\texttt{\textbf{procedure} \textrm{#1}(\param{#2})}~}
\newcommand*\pre[1]{\textit{\{ Vorbedingung: {#1} \}}}
\newcommand*\func[2]{\texttt{\textrm{#1}(\param{#2})}}
\newcommand*\proc[2]{\func{#1}{#2}}
\newcommand*\param[1]{\texttt{\emph{#1}}}
\newcommand*\outparam[1]{\textbf{out} #1}
\newcommand*\type[1]{\texttt{#1}}
\newcommand*\class[1]{\texttt{\textbf{class} #1}}
\newcommand*\const[1]{\textsc{#1}}
\newcommand*\switch[1]{\texttt{\textsc{#1}}}
\newcommand*\typedef[1]{\textbf{typedef}~#1;}
\newcommand*\struct[1]{\textbf{struct}~#1;}
\newcommand*\closefunc{\end{labeling} \textbf{end function}}
\newcommand*\closeproc{\end{labeling} \textbf{end procedure}}
\newcommand*\forlooptext[4]{\textbf{for} #1 \textbf{from} \(#2\) 
    \textbf{to} \(#3\) \textbf{step} \(#4\) \textbf{loop}}
\newcommand*\forloop[4]{\forlooptext{#1}{#2}{#3}{#4} 
    \begin{labeling}{ii} \item}
\newcommand*\closefortext{\textbf{end for}\\}
\newcommand*\closefor{\end{labeling} \closefortext}
\newcommand*\whileloop[1]{\textbf{while} \(#1\) \textbf{loop} 
    \begin{labeling}{ii} \item}
\newcommand*\closewhile{\end{labeling} \textbf{end for}\\}
\newcommand*\ifthen[1]{\textbf{if} \(#1\) \textbf{then} 
    \begin{labeling}{ii} \item}
\newcommand*\orelse[1]{~\mbox{\textbf{or else}}~#1}
\newcommand*\ifelse{\end{labeling} \textbf{else} \begin{labeling}{ii} \item}
\newcommand*\closeif{\end{labeling} \textbf{end if}\\}
\newcommand*\ret[1]{\textbf{return} #1\\}
\newcommand*\field[2]{\textbf{array} #1 \textbf{of} #2}

\newcommand*\integer{\textrm{integer}}
\newcommand*\bool{\textrm{boolean}}
\newcommand*\true{\textrm{true}}
\newcommand*\false{\textrm{false}}

%%%%%%%%%%%%%%%%%%%%%%%%%%%%%%%%%%%%%%%%%%%%%%%%%%%%%%%%%