% fazit.tex
% Einschaetzung der Arbeit und Ausblick
%
% Ausarbeitung zur Diplomarbeit Nr. 2035 - "Erzeugung und Evaluierung 
% von Oktalbaumstrukturen als Schnittstelle zu CAD-Programmen"
%
% Autor: Stefan Mahler 2002
%   Universitaet Stuttgart, SgS
% Betreuer: Ralf Mundani

\chapter{Fazit}
\label{fazit}
%Was sollte gemacht werden? Was wurde gemacht? Gut/ schlecht? 
%Verbesserungsmoeglichkeiten als Ausblick?
Wie \cite{diss_oct} zeigt k"onnen Oktalbaumstrukturen 
als integrierendes Element zwischen CAD-Programmen zur Geometriemodellierung 
und Simulation genutzt werden. Ziel dieser Arbeit war die Erzeugung und 
Evaluierung von Oktalbaumstrukturen als Schnittstelle zu CAD-Programmen. 
Im Mittelpunkt stand dabei die Generierung 
des Oktalbaums aus einem geeignenten weitverbreiteten frei zug"anglichen 
Oberfl"achenformat, in welches in der Abteilung befindliche 
Oberfl"achenmodelle leicht exportiert werden k"onnen. Neben der 
Unterst"utzung von triangulierten Modellen sollten auch Spline-Fl"achen 
verarbeitet werden k"onnen. Die Integrit"at des Oberfl"achenmodells soll
durch den Erzeuger des Modells gew"ahrleistet sein, eine "Uberpr"ufung 
innerhalb der in dieser Arbeit erstellten Implementierung ist somit nicht 
notwendig.

Es existieren bereits einige L"osungen, die in diese Richtung gehen und 
beachtenswerte Ergebnisse erzielen, wie \cite{dipl_anw_oct} oder
\cite{konvex_polygon_anw_oct}.
Diese L"osungen zeigen jedoch jeweils einige der folgenden
Unzul"anglichkeiten:
\begin{itemize}
\item Die L"osung ist nicht frei zug"anglich. Neben dem Code oder der 
    Architektur betrifft das z.T. die L"osungsidee oder die verwendeten
    Algorithmen.
\item Es kann der Oktalbaum nur aus einem Teil der geforderten m"oglichen 
    Oberfl"achenmodelle generiert werden, z.B. nur aus Polygonfl"achenmodellen 
    nicht aber f"ur K"orper mit Freiformfl"achen als Oberfl"ache.
\item Als Importformat f"ur die Oberfl"achenmodelle wird ein 'Spezialformat'
    verwendet, also keine weitverbreiteten Formate, deren Struktur frei 
    zug"anglich dokumentiert ist. Oberfl"achenmodelle in "ublichen Formaten 
    lassen sich dadurch nicht einfach in dieses Format exportieren, wodurch
    nicht die einschl"agigen Werkzeuge (CAD-Programme) zur Erzeugung der 
    Geometriemodelle verwendet werden k"onnen.
\item F"ur die Generierung der Oktalbaumstruktur aus dem Oberfl"achenmodell 
    werden Methoden und Makros eines Programms verwendet, dass zur Erstellung 
    und Bearbeitung von Oberfl"achenmodellen dient. Somit wird zur 
    Oktalbaumgenerierung stets das (nicht frei verf"ugbare) 
    Modellierungswerkzeug ben"otigt. Der Umfang von einsetzbaren 
    Datenstrukturen und verwendbaren Methoden ist stark durch den Umfang
    der durch das Modellierungswerkzeug bereitgestellter Bibliotheken
    korelliert, woraus sich auch weitere Erschwernisse bez"uglich 
    Erweiterbarkeit (insbesondere f"ur andere Formate) und Optimierbarkeit
    ergeben.
\end{itemize}
Durch die im Rahmen dieser Diplomarbeit erstellte Implementierung werden die 
Anforderungen erf"ullt. Sie stellt somit eine erfolgreiche Umsetzung der
Ziele dar und konnte damit die Nutzbarkeit von Oktalbaumstrukturen als 
Schnittstelle zu CAD-Programmen best"atigen. Dabei wurde u.a. bewusst ein 
anderer Weg gegen"uber dem klassischen Ansatz zur Oktalbaumerzeugung 
beschritten. "Ublicher Weise wird sukzessive die Lage des durch den aktuellen 
Oktalbaumknoten repr"asentierten W"urfels bez"uglich dem zu erzeugenden K"orper 
analysiert (in/out/on) und entsprechend unter Ausnutzung hierachischen 
Struktur des Oktalbaums verfeinert. Bei der hier erarbeiten Implementierung 
wird ein Index f"ur jeden Oktalbaumknoten definiert. Das Schema erlaubt mit 
Hilfe der Indizes einen zum Normzellenschema "aquivalenten Zugriff. Nun 
k"onnen wie bei einem Normzellenschema die Oberfl"achenw"urfel des K"orpers 
in den Oktalbaum eingef"ugt werden, anschlie"send werden die Innen- und 
Au"sengebiete durch ein spezielles Verfahren gef"ullt. Nach einer letztlichen 
Kompaktierung liegt der gew"unschte Oktalbaum vor. Die durchgef"uhrten 
Messungen bis zu einer Baumtiefe von $9$ zeigen: Die durch dieses Vorgehen 
erzielten Generierungsgeschwindigkeiten zeigen sehr gute Resultate. Durch die 
verwendete kompakte Datenstruktur ist nur ein relativ geringer Speicherplatz 
zur Generierung n"otig. 

Wird die Baumtiefe um $1$ erh"oht, steigen Laufzeit und Speicherbedarf um 
den Faktor $4$.
Die gew"ahlte Programmarchitektur erm"oglicht das 
einfache Erweitern des Programms zur Integration weiterer Formate und das 
Hinzuf"ugen alternativer bzw. das Erweitern vorhandener Algorithmen.
Hiermit 
lassen sich Werte auch f"ur gr"o"sere maximale Baumtiefen absch"atzen. 
Das Programm bietet jedoch keine M"oglichkeit bereits aus einem CAD-Modell 
erzeugte Oktalb"aume unter Nutzung des CAD-Modells weiter zu verfeinern.

Daraus lassen sich die Verbesserungsm"oglichkeiten ableiten: 
\begin{itemize}
\item Der erarbeitete Oktalbaumgenerator ist f"ur den Ein-Prozessor-Betrieb 
ausgelegt. Der 'klassische Algorithmus' zur Oktalbaumgenerierung nutzt die 
Organisationsprinzipien \emph{Hierachie}, \emph{Rekursion} und 
\emph{Adaptivit"at} aus und l"asst sich leicht in einen effizienten 
Algorithmus f"ur Mehrprozessorbetrieb umwandeln. Das hiervorgestellte 
Verfahren umgeht hingegen an einigen relevanten Stellen diese 
Organisationsprinzipien. So wird das F"ull-Verfahren auf dem unterliegenden 
virtuellen Gitter durchgef"uhrt, was eine effiziente Parallelisierung 
erschwert.
 
%- Generierung des Oktalbaums aus mehreren Quellen: bin"are Mengenoperationen
\item Eine n"utzliche Erweiterung w"are die Oktalbaumgenerierung unter Nutzung 
mehrerer Quellen, also der Import von Teilgeometrien aus unterschiedlichen 
Dateien. Ein m"ogliches Vorgehen hierf"ur w"are jeweils gesondert einen neuen 
Oktalbaum f"ur jede Datenquelle zu generieren und diesen mit den vorhandenen 
zu vereinigen. Es lassen sich einfach effiziente Algorithmen f"ur 
Mengenoperationen implementieren. Besondere Aufmerksamkeit verdient dabei 
jedoch die Kollisionsaufl"osung. Die zentrale Frage dabei ist, wie reagiert 
werden soll, wenn sich in der gleichen Zelle unterschiedliche K"orper 
befinden. Dabei m"ussen sich die realen K"orper im Gegensatz zum 
diskretisierten Oktalbaummodell gar nicht "uberschneiden.

%- Optimierungen Effizienz/Genauigkeit der Geometriewiedergabe bei einer 
%gew"ahlten Aufl"osung
%- andere Formate : CSG relativ einfach, erh"ohte Stellung bei der Integration
%von CAD-, Simulation und Visualisierung und den bei ihnen verwendeten 
%Geometriemodellen
\item Neben Optimierungen der Ausf"uhrungseffizenz des Programmcodes 
w"are die Unterst"utzung weiterer Formate eine sinnvolle Erweiterung. Die 
Architektur des Programms unterst"utzt das Hinzuf"ugen weiterer 
Import"=/Exportformate. Es k"onnte "uber den Einbau einer Importfunktion f"ur 
ein CSG-Schema nachgedacht werden: Mengenoperationen auf Oktalb"aumen sowie 
die Oktalbaumgenerierung f"ur Grundprimitive lassen sich sehr effizient 
realisieren. Da die Primitivenanzahl begrenzt und ihre Gestalt im Voraus 
bekannt sind, k"onnen zur Effizienzsteigerung relevante Daten zur 
Oktalbaumgenerierung sogar in einer Tabelle abgelegt werden.\footnote{"Ahnliche
Verfahren werden zur Quadratwurzelberechnung von Flie"skommazahlen verwendet.}
F"ur die Transformationsoperationen sind effiziente Verfahren bekannt 
\footnote{\cite{diss_oct} verweist in diesem Zusammenhang auf einen 
"Ubersichtsartikel von Chen und Huang.}
.

%- R"uckgewinnung der Ausgangsgeometrie: schwierig bis unm"oglich:
%    zus"atzliche Attributierung der Oktalbaumknoten n"otig,
%    Verlust an Strukturiertheit, Problem der Konsistenzhaltung des Oktalbaums
%    bei Modifikationen, Pr"amissen bei Effizienz, Modifikationsm"oglichkeiten,
%    Genauigkeit des r"uckgewonnenen Oberfl"achenmodells
\item Da die Diskretisierung, die bei der Oktalbaumgenerierung aus stetigen 
Modellen erfolgt, mit einem Informationsverlust einhergeht, ist eine 
R"uckgewinnung der Ausgangsgeometrie aus der Oktalbaumstruktur hingegen 
schwierig. Die Oktalbaumstruktur m"usste hierf"ur um zus"atzliche Attribute 
erweitert werden. 
Hieraus w"urde sich direkt ein Verlust der Strukturiertheit ergeben. Bei jeder 
Modifikation des Oktalbaums m"ussten die Attribute mit "uberarbeitet werden, 
um die Konsistenz der Oktalbaumstruktur zu erhalten. 
Pr"amissen bei Effizenz, Modifikationsm"oglichkeiten und Genauigkeit der 
r"uckgewonnen Geometrie w"aren unvermeidbar.
\end{itemize}

Dennoch zeigt sich die Eignung der Oktalbaumstruktur als Schnittstelle 
zur geometrischen Modellierung.

%% End of Document
